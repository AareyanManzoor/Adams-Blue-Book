\documentclass[../main]{subfiles}
\begin{document}

% walter

\chapter{Smash products}
\label{sec:p3c04}
    In this section we will construct smash products of spectra.
More precisely, we will construct from any two CW-spectra $X$ and $Y$ a CW-spectrum $X \wedge Y$, so as to have the properties stated in the following theorem, among other properties.
\begin{theorem}\label{thm:p3c04.1}
    \begin{enumerate}[label=(\alph*)] 
        \item $X \wedge Y$ is a functor of two variables, with arguments and values in the (graded) stable homotopy category.
        \item The smash-product is associative, commutative, and has the sphere-spectrum $S$ as a unit, up to coherent natural equivalences.
    \end{enumerate}
\end{theorem}

We explain that statement (a) is to be taken in the graded sense.
That is, if
\[
    f \in [X, X']_r, \quad g \in [Y, Y']_s
\]
then
\[
    f \wedge g \in [X \wedge Y, X' \wedge Y']_{r + s},
\]
and besides $1 \wedge 1 = 1$, we have
\[
    (f \wedge g) (h \wedge k) = (-1)^{bc} (fh) \wedge (gk)
\]
if $f \in [X', X'']_{a}$, $h \in [X, X']_b$, $g \in [Y', Y'']_c$, $k \in [Y, Y']_d$.

We explain statement (b).
It claims that there are the following equivalences in our category.

%% TODO
% Write equivalences for category(to be fixed in editing)
\[\begin{tikzcd}
	{a=} & {a(X,Y,Z)} {:(X\wedge Y)\wedge Z} & {X\wedge(Y\wedge Z)} \\
	{c=} & {C(X,Y)} {:X\wedge Y} & {Y\wedge X} \\
	{l=} & {l(Y)} {:S\wedge Y} & Y \\
	{r=} & {r(X)}  {:X\wedge S} & X
	\arrow[from=1-2, to=1-3]
	\arrow[from=2-2, to=2-3]
	\arrow[from=3-2, to=3-3]
	\arrow[from=4-2, to=4-3]
\end{tikzcd}\]

They are all of degree 0.
They are all natural as $X$, $Y$, and $Z$ vary over the stable category;
in the case of $c$ this means that the diagram

\[
\begin{tikzcd}
    X \wedge Y & X \wedge Y \\
    X' \wedge Y' & Y' \wedge X' \\
    \arrow[from=1-1, to=1-2, "c"]
    \arrow[from=1-1, to=2-1, "f \wedge g"]
    \arrow[from=1-2, to=2-2, "g \wedge f"]
    \arrow[from=2-1, to=2-2, "c"] 
\end{tikzcd}
\]
is commutative up to a sign $(-1)^{pq}$, if $f \in [X, X']_p$, $g \in [Y, Y']_q$.
The other naturality conditions are the obvious ones and don't involve signs.
The equivalences make the following diagrams commute in our category.
(If one thinks in terms of representative maps, one says that these diagrams are homotopy-commutative.)
%% TODO
% Add labels to next eight diagrams
\begin{enumerate}
    \item[(i)]\[
    \adjustbox{scale=0.8}{
    \begin{tikzcd}[column sep=-2em, row sep=3em] 
	&& {(W \wedge X) \wedge (Y \wedge Z)} \\
	{((W \wedge X) \wedge Y) \wedge Z} &&&& {W \wedge (X \wedge (Y \wedge Z))} \\
	& {(W \wedge (X \wedge Y)) \wedge Z} && {W \wedge ((X \wedge Y) \wedge Z)}
	\arrow["{a_2}", from=1-3, to=2-5]
	\arrow["{a_1}", from=2-1, to=1-3]
	\arrow["{a_3 \wedge 1}"', from=2-1, to=3-2]
	\arrow["{a_4}"', from=3-2, to=3-4]
	\arrow["{1 \wedge a_5}"', from=3-4, to=2-5]
    \end{tikzcd}
    }
    \]
    Here
    \begin{align*}
        a_1 &= a(W \wedge X, Y, Z) & a_4 &= a(W, X \wedge Y, Z) \\
        a_2 &= a(W, X, Y \wedge Z) & a_5 &= a(X, Y, Z) \\
        a_3 &= a(W, X, Y).
    \end{align*}
    \item[(ii)]\[
    \begin{tikzcd}[column sep=1.5em] 
       & Y \wedge X \\
        X \wedge Y && X \wedge Y
        \arrow[from=2-1, to=1-2, "c_1"]
        \arrow[from=2-1, to=2-3, "1"]
        \arrow[from=1-2, to=2-3, "c_2"] 
    \end{tikzcd}
    \]
    Here
    \begin{align*}
        c_1 &= c(X, Y) \\
        c_2 &= c(Y, X).
    \end{align*}
    \item[(ii)]\[
    \adjustbox{scale=1,center}{
    \begin{tikzcd}
    & (Y \wedge X) \wedge Z \ar[rd,"a"] & \\
    (X \wedge Y) \wedge Z \ar[ru,"c\wedge 1"] \ar[d,"a"] & & Y \wedge (X \wedge Z)\ar[d,"1\wedge c"] \\ 
    X \wedge (Y \wedge Z) \ar[rd,"c"] & & Y \wedge (Z \wedge X) \\ 
    & (Y \wedge Z) \wedge X \ar[ru,"a"] &
    \end{tikzcd}}\]
    
    Here the morphisms can be made more precise, as in $(i)$ and $(ii)$
    \item[(iv)] \[\begin{tikzcd}
	{(S\wedge Y)\wedge Z} && {S\wedge(Y\wedge Z)} \\
	& {Y\wedge Z}
	\arrow["a", from=1-1, to=1-3]
	\arrow["l"', from=1-1, to=2-2]
	\arrow["l\wedge 1", from=1-3, to=2-2]
    \end{tikzcd}\]
    \item[(v)]\[\begin{tikzcd}
	{(X\wedge S)\wedge Z} && {X\wedge(S\wedge Z)} \\
	& {X\wedge Z}
	\arrow["a", from=1-1, to=1-3]
	\arrow["1\wedge l"', from=1-1, to=2-2]
	\arrow["r\wedge 1", from=1-3, to=2-2]
    \end{tikzcd}\]
    \item[(vi)]\[\begin{tikzcd}
	{(X\wedge Y)\wedge S} && {X\wedge(Y\wedge S)} \\
	& {X\wedge Y}
	\arrow["a", from=1-1, to=1-3]
	\arrow["1\wedge r"', from=1-1, to=2-2]
	\arrow["r", from=1-3, to=2-2]
    \end{tikzcd}\]
    \item[(vii)]\[\begin{tikzcd}
	{S\wedge Y} && {Y\wedge S} && {X\wedge S} && {S\wedge X} \\
	& Y &&&& X
	\arrow["c", from=1-1, to=1-3]
	\arrow["l"', from=1-1, to=2-2]
	\arrow["r", from=1-3, to=2-2]
	\arrow["c", from=1-5, to=1-7]
	\arrow["r"', from=1-5, to=2-6]
	\arrow["l", from=1-7, to=2-6]
    \end{tikzcd}\]
    
    (These are equivalent, in view of $(iii)$)
    \item[(viii)] \[\begin{tikzcd}
	{S\wedge S} && {S\wedge S}
	\arrow["c", bend left=30, from=1-1, to=1-3]
	\arrow["1"', bend right=30, from=1-1, to=1-3]
    \end{tikzcd}\]
\end{enumerate}



It follows from these properties that every other diagram constructed from $a$, $c$, $\ell$, and $r$ which you might conceivably wish to prove commutative, is commutative; see MacLane \cite{maclane}.

The properties stated in this theorem are not intended to be a complete list.
We also want our smash-products to be compatible with those which we already have for CW-complexes.
We can take it as a guiding idea that if $X$ is a CW-spectrum with terms $X_n$, and $Y$ is a CW-spectrum with terms $Y_m$, then we want $X \wedge Y$ to be the thing to which $X_n \wedge Y_m$ tends as $n$ and $m$ tend to infinity.
It is therefore tempting to define a product spectrum $P$ so that
\[
    P_p = X_{n(p)} \wedge Y_{m(p)},
\]
where $n(p)$ and $m(p)$ are fixed functions such that $n(p) + m(p) = p$, while $n(p) \to \infty$ and $m(p) \to \infty$.
This approach gives the ``handicrafted smash products'' (in later versions, ``naive smash products'') of Boardman.
Of course, there are many different ways of choosing the function $n(p)$ and $m(p)$, and these give rise to different ``handicrafted smash products'';
it is obviously desirable to prove that these different products are related by natural equivalences.
For later work it is also desirable to have a notation more convenient than that of functions $n(p)$ and $m(p)$;
it is for this purpose that we introduce the details which follow next.

Let $A$ be an ordered set, isomorphic to the ordered set $\{0, 1, 2, 3, \ldots\}$.
(The reason that we do not take $A$ to be the ordered set $\{0, 1, 2, 3, \ldots\}$ is that we will later want to take $A$ to be a subset of $\{0, 1, 2, 3, \ldots\}$.)
Let $B$ be a subset of $A$;
then we define a corresponding function
\[
    \beta: A \longrightarrow \{0, 1, 2, 3, \ldots\}
\]
as follows:
$\beta(a)$ is the number of elements $b \in B$ such that $b < a$.
Then $\beta$ is monotonic, and $\beta |_B$ is an order-preserving isomorphism between $B$ and some initial segment of $\{0, 1, 2, 3, \ldots\}$.
The notation $\beta$ emphasizes the dependence of $\beta$ on $B$ rather than on $A$;
this is legitimate, for if we have $B \subset A \subset A'$, then the function $\beta_A$ defined on $A$ is the restriction to $A$ of the function $\beta_{A'}$ defined in $A'$.

Next suppose given a partition of $A$ into two subsets $B$ and $C$, so that $A = B \cup C$, $B \cap C = \emptyset$.
A suitable illustration is obtained by taking
    \begin{align*}
        A &= \{0, 1, 2, 3, \ldots\} \\
        B &= \{0, 2, 4, 6, \ldots \} \\
        C &= \{1, 3, 5, 7, \ldots\}
    \end{align*}
but there are many other equally suitable choices.
Then we define a smash-product functor which assigns to any two CW-spectra $X$ and $Y$ a CW-spectrum $X \wedge_{BC} Y$.
It is convenient to display only $B$ and $C$ in the notation, but of course the product depends on the ordering of $B \cup C$.

The terms of the product spectrum
\[
    P = X \wedge_{BC} Y
\]
are given by
\[
    P_{\alpha(a)} = X_{\beta(a)} \wedge Y_{\gamma(a)}.
\]
Note that $\alpha$ is an isomorphism from the ordered set $A = B \cup C$ to $\{0, 1, 2, 3, \ldots\}$ and $\beta$, $\gamma$ are monotonic functions from $A = B \cup C$ to the set $\{0, 1, 2, 3, \ldots\}$ such that $\beta(a) + \gamma(a) = \alpha(a)$.

The maps of the product spectrum are defined as follows.
We have
\[
    P_{\alpha(a)} \wedge S^{1} \longrightarrow X_{\beta(a)} \wedge Y_{\gamma(a)} \wedge S^{1}.
\]
Here it is convenient to regard $S^{1}$ as $\mathbb{R}^{1}$ compactified by adding a point at infinity, which becomes the base-point.
This allows us to define a map of degree $-1$ from $S^{1}$ to $S^{1}$ by $t \mapsto -t$.

If $a \in B$, then
\[
    P_{\alpha(a) + 1} = X_{\beta(a) + 1} \wedge Y_{\gamma(a)}
\]
and we define the map
\[
    \pi_{\alpha(a)} \colon SP_{\alpha(a)} \longrightarrow P_{\alpha(a) + 1}
\]
by 

\[
    \pi_{\alpha(a)}(x \wedge y \wedge t) = \xi_{\beta(a)} \big(x \wedge (-1)^{\gamma(a)}t\big) \wedge y
\]
If $a\in C$, then
\[
    P_{\alpha(a)+1}=X_{\beta(a)}\wedge Y_{\gamma(a)+1}
\]
and we define the map
\[
    \pi_{\alpha(a)}(x \wedge y \wedge t) = x \wedge \eta_{\gamma(a)} (y \wedge t).
\]
Here
\[
    x \in X_{\beta(a)}, \quad y \in Y_{\gamma(a)}, \quad t \in S^{1},
\]
and
\[
    \xi_{\beta(a)} \colon X_{\beta(a)} \wedge S^{1} \longrightarrow X_{\beta(a)}, \qquad \eta_{\gamma(a)} \colon Y_{\gamma(a)} \wedge S^{1} \longrightarrow Y_{\alpha(a) + 1}
\]
are the appropriate maps from the spectra $X$, $Y$.
The sign $(-1)^{\gamma(a)}$ is introduced, of course, because we have moved $S^{1}$ across $Y_{\gamma(a)}$.

It is clear that $P = X \wedge_{BC} Y$ is functorial for functions of $X$ and $Y$ of degree 0.
Next we point out that we have no assumed that the sets $B$ and $C$ are infinite.
In the obvious applications they are infinite, so that $\beta(a) \longrightarrow \infty$ and $\gamma(a) \longrightarrow \infty$;
but it is convenient to allow $B$ and $C$ to be finite.
For example, let $\underline{S}^{1}$ be the suspension spectrum of $S^{1}$;
then $\underline{S}^{1} \wedge_{\emptyset, A} Y = \mathrm{Susp}(Y)$.
If $B$ is infinite, and $X'$ is a cofinal subspectrum of $X$, then $X' \wedge_{BC} Y$ is a cofinal subspectrum $X \wedge_{BC} Y$.
So in this case $X \wedge_{BC} Y$ is natural for maps of $Y$ of degree 0.
Next we observe that $(\te{Cyl}(X)) \wedge_{BC} Y$ and $X \wedge_{BC} (\te{Cyl}(Y))$ can be identified with $\te{Cyl}(X \wedge_{BC} Y)$.
It follows that the homotopy class of $f \wedge_{BC} g$ depends only on the homotopy class of $f$ (if $B$ is infinite) or $g$ (if $C$ is infinite).

We propose to construct $X \wedge Y$ to have the properties stated in the following theorem.
\begin{theorem}\label{thm:p3ch04.2}
    For each choice of $B$, $C$ there is a morphism
    \[
        \mathrm{eq}_{BC} \colon X \wedge_{BC} Y \longrightarrow X \wedge Y \quad \text{(of degree 0)}
    \]
    with the following properties.
    \begin{enumerate}[label=(\roman*)] 
        \item If $B$ is infinite and $f \colon X \longrightarrow X'$ is a morphism of degree 0, then the following diagram is commutative.
            \[
            \begin{tikzcd}[column sep=5em] 
                X \wedge_{BC} Y & X \wedge Y \\
                X' \wedge_{BC} Y & X' \wedge Y
                \arrow[from=1-1, to=1-2, "\mathrm{eq}_{BC}"]
                \arrow[from=1-1, to=2-1, "f \wedge_{BC} 1"'] 
                \arrow[from=1-2, to=2-2, "f \wedge 1"]
                \arrow[from=2-1, to=2-2, "\mathrm{eq}_{BC}"] 
            \end{tikzcd}
            \]
        \item If $C$ is infinite and $g: Y \longrightarrow Y'$ is a morphism of degree 0, then the following diagram is commutative.
            \[
            \begin{tikzcd}[column sep=5em] 
                X \wedge_{BC} Y & X \wedge Y \\
                X \wedge_{BC} Y' & X \wedge Y'
                \arrow[from=1-1, to=1-2, "\mathrm{eq}_{BC}"]
                \arrow[from=1-1, to=2-1, "1 \wedge_{BC} g"'] 
                \arrow[from=1-2, to=2-2, "1 \wedge g"]
                \arrow[from=2-1, to=2-2, "\mathrm{eq}_{BC}"] 
            \end{tikzcd}
            \]
        \item The morphism $\mathrm{eq}_{BC} \colon: X \wedge_{BC} Y \longrightarrow X \wedge Y$ is an equivalence if any one of the following conditions is satisfied.
            \begin{enumerate}[label=(\alph*)] 
                \item $B$ and $C$ are infinite.
                \item $B$ is finite, say with $d$ elements, and $\xi_r \colon \Sigma X_r \longrightarrow X_{r+1}$ is an isomorphism for $r \geq d$.
                \item $C$ is finite, say with $d$ elements, and $\eta_r \colon \Sigma Y_r \longrightarrow Y_{r+1}$ is an isomorphism for $r \geq d$.
            \end{enumerate}
    \end{enumerate}
\end{theorem}

Let me show how Theorem \ref{thm:p3ch04.2} will help to prove Theorem \ref{thm:p3c04.1}(b).
Consider first the associativity.
The point is that the ``handicrafted smash products'' are actually associative if you pick the right product at each point.
More precisely, take a set $A$ and partition it into three disjoint subsets $B$, $C$, and $D$, such that $B \cup C$ and $C \cup D$ are infinite.
Let $X$, $Y$, and $Z$ be CW-spectra.
Then we can form the spectra
\[
    (X \wedge_{BC} Y) \wedge_{B \cup C, D} Z \quad \text{and} \quad X \wedge_{B, C \cup D} (Y \wedge_{CD} Z).
\]
(Now one begins to see the purpose for which the notation was designed.)
These two spectra are actually the same spectrum.
For the terms of each are given by
\[
    P_{\alpha(a)} = X_{\beta(a)} \wedge Y_{\gamma(a)} \wedge Z_{\delta(a)}.
\]
The maps of each are described in the same way as before.
We have
\[
    \Sigma P_{\alpha(a)} = X_{\beta(a)} \wedge Y_{\gamma(a)} \wedge Z_{\delta(a)} \wedge S^{1}.
\]
If $a\in B$, then 
\[ P_{\alpha(a)+1} = Z_{\beta(a)+1} \wedge Y_{\gamma(a)} \wedge Z_{\delta(a)}\]
and we have 
\[ \pi_{\alpha(a)}(x \wedge y \wedge z\wedge t) = \xi_{\beta(a)} \big(x \wedge (-1)^{\gamma(a)+\delta(a)}t \big)\wedge y\wedge z\]
If $a\in C$ then 
\[P_{\alpha(a)+1} = X_{\beta(a)} \wedge Y_{\gamma(a)+1} \wedge Z_{\delta(a)}\]
and we have 
\[
\pi_{\alpha(a)}(x\wedge y\wedge z\wedge t) = x \wedge \eta_{\gamma(a)}\big(y\wedge (-1)^{\delta(a)} t\big) \wedge z
\]
If $a\in D$ then
\[
P_{\alpha(a)+1} = X_{\beta(a)} \wedge Y_{\gamma(a)} \wedge Z_{\delta(a)+!}
\]
and we have 
\[\pi_{\alpha(a)}(x \wedge y\wedge z \wedge t) = x \wedge y\wedge  \zeta_{\delta(a)} (z \wedge t)\]
Here, of course, we have $x\in X_{\beta(a)}$, $y\in Y_{\gamma(a)}$, $z\in Z_{\delta(a)}$, $t\in S^1$ and $\xi_{\beta(a)},\eta_{\gamma(a)}, \zeta_{\delta(a)}$ are the appropriate maps of the spectra $X,Y,Z$.
We will arroung our construction to thave the following property.
\begin{theorem}\label{thm:p3c04.3}
    The equivalence
    \[a=a(X,Y,Z):(X\wedge Y)\wedge Z\lar{} X\wedge(Y\wedge Z)\]
    makes the following diagram commutative for each choice of $B,C,D$
    %big diagram here
    \[
    \adjustbox{scale = 0.8, center} {
    \begin{tikzcd}[column sep=0.4cm, row sep = 0.8cm]
     & {(X \wedge Y) \wedge Z} \arrow[rrr,"a"] & & & {X \wedge (Y \wedge Z)} \\
    {(X \wedge Y) \wedge_{B \cup C, D}Z} \arrow[ur,"eq_{B\cup C,D}"] & & {(X \wedge_{BC}Y)\wedge Z} \arrow[ul,"eq_{BC}\wedge 1"] & {X \wedge(Y\wedge_{C,D}Z)} \arrow[ur,"1\wedge eq_{CD}"] & & {X \wedge_{B,C\cup D}(Y \wedge Z)} \arrow[ul, "eq_{B,C\cup D}"]\\
     & {(X \wedge_{BC}Y)\wedge_{B\cup C, D}Z} \arrow[rrr,"1"] \arrow[ul,"eq_{B,C}^{B\cup C, D}1"] \arrow[ur,"eq_{B\cup C,D}"] & & & {X \wedge{B,C\cup D}(Y \wedge_{CD} Z)} \arrow[ul,"eq_{B,C\cup D}"] \arrow[ur,"1\wedge_{B,C\cup D}eq_{CD}"] \\
    \end{tikzcd}
    }\]
\end{theorem}

Note that the squares are commutative by the naturality of $\te{eq}$; we can apply \ref{thm:p3ch04.2}(i) and (ii) since $B\cup C$ and $C\cup D$ are infinite.

Let us now show how to check the commutativity of diagram $(i)$ in Theorem \ref{thm:p3c04.1}(b) (the pentagon diagram). Then by Theorems \ref{thm:p3ch04.2} and \ref{thm:p3c04.3},all we have to do is check the following diagram is commutative.
\[
\adjustbox{scale=0.8, center}{
\begin{tikzcd}
    & {(W\wedge_{BC}X)\wedge_{B\cup C,D\cup E}(Y\wedge_{DE}Z)} \\
    \\
    {((W\wedge_{BC}X)\wedge_{B\cup C,D}Y)\wedge_{B\cup C\cup D,E}Z} && {W\wedge_{B,C\cup D\cup E}(X\wedge_{C,D\cup E}(Y\wedge_{DE}Z))} \\
    \\
    {(W\wedge_{B,C\cup D}(X\wedge_{CD}Y))\wedge_{B\cup C\cup D,E}Z} && {W\wedge_{B,C\cup D\cup E}((X\wedge_{CD}Y)\wedge_{C\cup D,E}Z)}
    \arrow["1", from=3-1, to=1-2]
    \arrow["1", from=1-2, to=3-3]
    \arrow["1"', from=3-1, to=5-1]
    \arrow["1"', from=5-1, to=5-3]
    \arrow["1"', from=5-3, to=3-3]
\end{tikzcd}}\]
This diagram is commutative as a diagram of functions before we pass to homotopy classes.

Similarly, the ``handicrafted smash products'' are commutative if
you pick the right product at each point. It is tempting to partition $A$ as $B\cup C$, and consider $X\wedge_{BC}Y$ and $Y\wedge_{CB}X$. Corresponding terms
of these spectra are isomorphic; it is tempting to define 
\[c_{\alpha(a)} :X_{\beta(a)}\wedge Y_{\gamma(a)}\lar{} Y_{\gamma(a)}\wedge X_{\beta(a)}\]
by \[c_{\alpha(a)} (x\wedge y)=y\wedge x\]
However, these components do not give a function between spectra, because the relevant diagrams do not commute. We should have inserted a sign  $(-1)^{\beta(a)\gamma(a)}$, and we do not have a spare suspension coordinate to
reverse. The answer is easy; we need only consider partitions $A=B\cup C$ such that $ \beta(a)\gamma(a)$ is always even.This amounts to the
following condition, Elements number $0$ and $1$ in $A$ must be either two
 elements of $B$, or else two elements of $C$. Similarly for elements
number $2$ and $3$, and similarly for elements number $2r$ and $2r+1$ for
each $r$.

Now that we realize we can restrict the choice of partition in this way,
we see that it is easy and useful to go further. In fact, we now introduce
the following restriction on the partition $A = B\cup C$.
\begin{condition}\label{con:p3ch04.4}
Elements number $0,1,2$, and $3$ in $A$ are either
four elements of $B$, or else four elements of $C$; similarly for elements
number $4,5,6$ and $7$ in $A$, and similarly for elements number $4r$,
$4e+1, 4r+2$ and $4r+3$ for each $r$. 
\end{condition}
With this restriction, we define an isomorphism 
\[ c=c_{BC}:X\wedge_{BC}Y \lar{} Y\wedge_{CB}X\]
in the manner suggested:
\[c_n(x\wedge y)= y\wedge x\]
This is clearly natural for functions of $X$ and $Y$. it is also natural for maps of $X$ if $B$ is infinite; similarly for $Y$ if $C$ is infinite.

We will arrange our constructions to have the following property.
\begin{theorem}\label{thm:p3ch04.5}
The equivalence $c=c(X,Y):X\wedge Y\lar{}Y\wedge X$ makes the following diagram commutative for each choice of $B,C$ satisfying \ref{con:p3ch04.4}
\[\begin{tikzcd}
X\wedge Y \arrow[rr,"c"]&& Y\wedge X\\ X\wedge_{BC} Y\arrow[u,"\te{eq}_{BC}"]\arrow[rr,"c_{BC}"]&&Y\wedge_{CB}X\arrow[u,"\te{eq}_{CB}"]
\end{tikzcd}\]
    
\end{theorem}

Let us now show how to check the commutativity of diagram $(iii)$ in Theorem \ref{thm:p3c04.1}(b) (the hexagon diagram). Take a set $A$ and partition it into three infinite subsets $B,C$ and $D$ satisfying the obvious analogue of condition \ref{con:p3ch04.4}. Then by Theorems \ref{thm:p3ch04.2}\ref{thm:p3c04.3}\ref{thm:p3ch04.5}, all we have to do is to check that the following diagram is commutative.
\[
\adjustbox{scale=1, center}{
\begin{tikzcd}
    & {(Y\wedge_{C,B}X)\wedge_{C\cup B,D}Z} \\
    {(X\wedge_{BC}Y)\wedge_{B\cup C,D}Z} && {Y\wedge_{C,B\cup D}(X\wedge_{B,D}Z)} \\
    {X\wedge_{B,C\cup D}(Y\wedge_{C,D}Z)} && {Y\wedge_{C,D\cup B}(Z\wedge_{D,B}X)} \\
    & {(Y\wedge_{C,D}Z)\wedge_{C\cup D,B}Z}
    \arrow["c", from=3-1, to=4-2]
    \arrow["1", from=4-2, to=3-3]
    \arrow["{1\wedge_{C,D\cup B}c}"', from=2-3, to=3-3]
    \arrow["1"', from=1-2, to=2-3]
    \arrow["1", from=2-1, to=3-1]
    \arrow["{c\wedge_{B\cup C,D}1}"', from=2-1, to=1-2]
\end{tikzcd}}\]
This diagram is commutative as a diagram of functions.

Similarly, suppose we wish to check the commutativity of diagram $(ii)$ in Theorem \ref{thm:p3c04.1}(b). By theorems \ref{thm:p3ch04.2} and \ref{thm:p3ch04.5}, all we have to do is check that the following diagram is commutative.

\[\begin{tikzcd}
& Y\wedge_{CB} X\arrow[rd,"c_{BC}"] & \\
X\wedge_{BC} Y \arrow[ru,"C_{BC}"] \arrow[rr, "1"] && X\wedge_{BC}Y
\end{tikzcd}\]
This diagram, too, is commutative as a diagram of functions.

Similarly, the “handicrafted smash products" have $S$ as a unit if
you pick the right product at each point. More precisely, suppose we
partition $A$ as $\emptyset\cup A$.This is a legitimate partition satisfying the
condition \ref{con:p3ch04.4}; this was the reason that we allowed the set $B$ to be finite.We can form the spectrum $S\wedge_{\emptyset A}Y$ and it is isomorphic to $Y$; the obvious isomorphism has as its components the isomorphisms $S^0\wedge Y_n \cong Y_n$. This isomorphism is natural for morphisms of degree $0$. We can now define \[ l:S\wedge Y\lar{}Y\]
to be the composite \[\begin{tikzcd}
S\wedge Y &&S\wedge_{\emptyset A}Y\cong Y\arrow[ll,"\te{eq}_{\emptyset,A}"]
\end{tikzcd}\]
Here $\te{eq}_{\emptyset A}$ is an equivalence by \ref{thm:p3ch04.2}(iii)(b). Similarly, we can form the spectrum $X\wedge_{A\emptyset} S$, and it is isomorphic to $X$; the obvious isomoprhism has as its components the isomorphisms $X_n\wedge S^0 \cong X_n$. As before, this isomorphism is natural for morphisms of degree $0$. We now define \[ r:X\wedge S\lar{}X\]
to be the composite \[\begin{tikzcd}
X\wedge S &&X\wedge_{A\emptyset }S\cong X\arrow[ll,"\te{eq}_{A,\emptyset}"]
\end{tikzcd} \]
Here $\te{eq}_{A\emptyset}$ is an equivalence by \ref{thm:p3ch04.2}(iii)(c).

To check the commutativity of diagrams (iv), (v), (vi) and (vii) in
Theorem \ref{thm:p3c04.1}(b), we have only to check that the following diagrams are
commutative.
%diagrams
\[\begin{tikzcd}
    {(S\wedge_{\emptyset B}Y)\wedge_{BC}Z} && {S\wedge_{\emptyset,B\cup C}(Y\wedge_{BC}Z)} \\
    & {Y\wedge_{BC}Z}
    \arrow["\cong", from=1-3, to=2-2]
    \arrow["1", from=1-1, to=1-3]
    \arrow["{\cong\wedge_{BC}1}"', from=1-1, to=2-2]
\end{tikzcd}\]
\[\begin{tikzcd}
    {(X\wedge_{B\emptyset}S)\wedge_{BC}Z} && {X\wedge_{BC}(S\wedge_{\emptyset C}Z)} \\
    & {X\wedge_{BC}Z}
    \arrow["{1\wedge_{BC}\cong}", from=1-3, to=2-2]
    \arrow["1", from=1-1, to=1-3]
    \arrow["{\cong\wedge_{BC}1}"', from=1-1, to=2-2]
\end{tikzcd}\]
\[\begin{tikzcd}
    {(X\wedge_{BC}Y)\wedge_{B\cup C,\emptyset}S} && {X\wedge_{BC}(Y\wedge_{C\emptyset}S)} \\
    & {X\wedge_{BC}Y}
    \arrow["{1\wedge_{BC}\cong}", from=1-3, to=2-2]
    \arrow["\cong"', from=1-1, to=2-2]
    \arrow[from=1-1, to=1-3]
\end{tikzcd}\]
\[\begin{tikzcd}
    {S\wedge_{\emptyset A}Y} && {Y\wedge_{A\emptyset}S} \\
    & Y
    \arrow["\cong", from=1-3, to=2-2]
    \arrow["\cong"', from=1-1, to=2-2]
    \arrow["c", from=1-1, to=1-3]
\end{tikzcd}\]
\[\begin{tikzcd}
    {X\wedge_{A\emptyset}S} && {S\wedge_{\emptyset A}X} \\
    & X
    \arrow["\cong", from=1-3, to=2-2]
    \arrow["\cong"', from=1-1, to=2-2]
    \arrow["c", from=1-1, to=1-3]
\end{tikzcd}\]
These diagrams are all commutative as diagrams of functions. 

Finally, we comment on part (viii) of Theorem \ref{thm:p3c04.1}(b). If you believe any of these results you must believe that $S\wedge S$ is equivalent to $S$. So $[S\wedge S,S\wedge S]_0 \cong [S,S]_0 = \bbZ$. So all we have to do is check that $c\colon S\wedge S \lra S\wedge S$ has degree 1; but we shall make our constructions to have the obvious effect on orientations.

We now turn to the constructions necessary to prove Theorems \ref{thm:p3c04.1},
\ref{thm:p3ch04.2}, \ref{thm:p3c04.3} and \ref{thm:p3ch04.5}. First we give a simple construction which is used in
proving Theorem \ref{thm:p3ch04.2}; this is the telescope functor. If $f_n:X_n\lar{}Y_n$
is a sequence of maps of CW-complexes, we can form the iterated
mapping cylinder, or telescope. If the are taken to be cellular, the
telescope is a CW-complex. We apply this construction to the terms of
a Spectrum of certain form, Let $X$ be a spectrum consisting of CW-complexes $X_n$ with base-point and cellular maps $\xi_n:X_n\wedge S^1\lar{}X_{n+1} ;$
 we need not even assume that $\xi_n$ is an isomorphism from $X_n\wedge S^1$ to a
subcomplex of $X_{n+1}$; the telescope functor $\te{Tel}$ \index{Telescope functor} will convert a spectrum
$X$ which does not have this property into one which does. 

We take the half-line $i\geq 0$ and divide it into $0$-cells $[i]$ and $1$-cells $[i,i+1]$ for $i=0,1,2,\dots$. We define the n$\nth$ term of $\te{Tel}(X)$ as a quotient space of the following wedge-sum
\[\bigg(\bigvee_{i=0}^{n-1} [i,i+1]^+\wedge X_i\wedge S^{n-i} \bigg)\vee \bigg(\bigvee_{i=0}^n [i]^+\wedge X_i\wedge S^{n-i}\bigg) \]
Here it is convenient to regard $S^m$ as $\bR^m$ compactified by adding point at infinity, which becomes the base-point. In this way the isomorphism $\bR^m\times \bR^n \lar{} \bR^{m+n}$ gives an isomorphism $S^m\wedge S^n \lar{} S^{m+n}$ which is convenient for later use. The following identifications are to
be made. Identify the point 
\[i\wedge x\wedge t\in [i,i+1]^+ \wedge X_i \wedge S^{n-i} \]
with the point \[i\wedge x\wedge t \in [i]^+\wedge X_i\wedge S^{n-i} \]
Identify the point \[(i+1)\wedge x\wedge t\wedge u\in [i]^+\wedge X_i\wedge S^{n-i-1}\]
with
\[(i+1)\wedge \xi_i(x\wedge t)\wedge u\in [i]^+\wedge X_i\wedge S^{n-i-1}\]
We give $\te{Tel}(X)_n$ the obvious structure as a CW-complex.

The n$\nth$ map of the spectrum $\te{Tel}(X)$ is obtained by passing to quotients from the obvious isomorphism of \[\bigg\{\bigg(\bigvee_{i=0}^n [i]^+\wedge X_i\wedge S^{n-i}\bigg)\vee \bigg(\bigvee_{i=0}^{n-1} [i,i+1]^+\wedge X_i\wedge S^{n-i} \bigg) \bigg\}\wedge S^1 \]
with 
\[\bigg(\bigvee_{i=0}^n [i]^+\wedge X_i\wedge S^{n-i+1}\bigg)\vee \bigg(\bigvee_{i=0}^{n-1} [i,i+1]^+\wedge X_i\wedge S^{n-i+1} \bigg).\]

There is an obvious homotopy equivalence $r_n:\te{Tel}(X)_n\lar{}X_n$ (collapse the telescope to its right-hand end $[n]^+\wedge X_n\wedge S^0$). These equivalences give the components of a function $r:\te{Tel}(X)\lar{}X$. This function is a weak equivalence, by \ref{lem:p3ch03.4}.

We pause to observe that this construction is functorial. It is clear
that a function $f:X\lar{}Y$ induces $\te{Tel}(f):\te{Tel}(X)\lar{}\te{Tel}(Y)$. If $X'$ is a subspectrum of $X$,then $\te{Tel}(X')$ is a subspectrum of $\te{Tel}(X)$. Unfortunately, if $X$ is a CW-spectrum and $X'$ is cofinal in $X$, it does not follow that $\te{Tel}(X')$ is cofinal in $\te{Tel}(X)$. So we avoid saying that a map of $X$ induces a map of $\te{Tel}(X)$. However, the injection $\te{Tel}(X')\lar{} \te{Tel}(X) $ is a homotopy equivalence, as we see using Theorem \ref{cor:p3ch03.5}. Morever, we can identify $\te{Tel}(\te{Cyl}(X))$ with $\te{Cyl}(\te{Tel}(X))$. It follows that a homotopy class of maps of $X$ induces a homotopy class of
maps of $\te{Tel}(X)$. We can now remark that $r$ is a natural transformation,
These facts are, of course, fairly trivial, but we need to cite this
passage later; it is for this reason that I have avoided a short~cut--one
could define $\te{Tel}$ on morphisms by requiring that $r$ be natural. 

We propose to arrange for Theorem \ref{thm:p3ch04.2} to be true by constructing $X\wedge Y$ so that it contains a copy of $\te{Tel}(X\wedge_{BC}Y)$ for each choice of $B$ and $C$. The morphism \[eq_{BC}:X\wedge_{BC}Y\lar{}X\wedge Y\]
Will be defined as the following composite:
\[X\wedge_{BC}Y\lal{r}\te{Tel}(X\wedge_{BC}Y)\lar{}X\wedge Y\]

The construction of $X\wedge Y$ (call it $P$) is as a "double telescope": That is, just as the parts of $\te{Tel}(X)$ corresponded to the cells of a cell-decomposition of the half-line $i\geq 0$, so here we make a similar use of the quarter-plane $i\geq 0,j\geq 0$. We divide the half-line $i\geq 0$ with $0$-cells $[i]$ and $1$-cells $[i,i+1]$, $i=0,1,2,\dots$. We divide the half-line $j\geq 0$ similarly, and we divide the quarter plane $i\geq 0,j\geq 0$ into the products of these cells. Thus we have four cells $e_{ij}$ with bottom left-hand corner at $(i,j)$:
\begin{align*}
    \text{the 0-cell } & [i]\times [j]\\
    \text{the 1-cells }&[i,i+1]\times [j]\text{ and }[i]\times [j,j+1]\\
    \text{the 2-cell }&[i,i+1]\times [j,j+1] 
\end{align*}

To construct $P_n$ we use those cells $e_{ij}$ which lie entirely in the part of the quarter-plane given by $x+y\leq n$. The condition for this is
\[i+j+\dim(e_{ij})\leq n\]
Let us start from
\[\bigvee_{i+j\leq n}([i]\times [j])^+ \wedge X_i \wedge Y_j \wedge S^{n-i-j}\]
and attach
\[\bigvee e_{ij}^+ \wedge X_i\wedge Y_j\wedge S^1 \wedge S^{n-i-j}\]
where $e_{ij}$ runs over the $1$-cells $[i,i+1]\times [j]$ and $[i]\times [j,j+1]$ such that $i+j+1\leq n$. The identifications are obvious. The point
\[(i,j)\wedge x\wedge y\wedge s\wedge t \text{  in  }e_{ij}^+\wedge X_i \wedge Y_j \wedge S^1 \wedge S^{n-i-j-1} \]
is to be identified with
\[(i,j)\wedge x\wedge y\wedge (s\wedge t) \text{  in  }([i]\times [j])^+\wedge X_i \wedge Y_j \wedge S^{n-i-j} \]
The point
\[(i+1,j)\wedge x\wedge y\wedge s\wedge t \text{  in  }([i+1]\times [j])^+\wedge X_{i} \wedge Y_j\wedge S^1 \wedge S^{n-i-j-1} \]
is to be identified with
\[(i+1,j)\wedge \xi_i\big( x\wedge (-1)^j s\big)\wedge y\wedge t \text{  in  }([i+1]\times [j])^+\wedge X_{i+1} \wedge Y_j \wedge S^{n-i-j-1} \]
The point
\[(i,j+1)\wedge x\wedge y\wedge s\wedge t \text{  in  }([i]\times [j,j+1])^+\wedge X_{i} \wedge Y_j\wedge S^1 \wedge S^{n-i-j-1} \]
is to be identified with
\[(i,j+1)\wedge x\wedge \eta_j(y\wedge s)\wedge t \text{  in  }([i]\times [j+1])^+\wedge X_{i} \wedge Y_{j+1} \wedge S^{n-i-j-1} \]

Consider now a cell $e=[i,i+1]\times [j,j+1]$ such that $i+j+2\leq n$. We have just described the subcomplex of $P_n$ corresponding to $\partial e$. Morever, it contains a family of subspaces $X_i\wedge Y_j \wedge S^2 \wedge S^{n-i-j-2}$, parametrized by the points of $\partial e$. Unfortunately, this family is not a proudct family, at least, not in a completely trivial way. Lets start from the point
\[(i,j)\wedge x\wedge y\wedge s\wedge t\wedge u \text{  in  }([i]\times[j])^+\wedge X_i \wedge Y_j \wedge S^1 \wedge S^1\wedge S^{n-i-j-1} \]
if we first increase $i$ and then increase $j$, we get first to \newline $(i+1,j)\wedge \xi_i\big(x\wedge (-1)^j s\big) \wedge y\wedge t\wedge u $ and then to $(i+1,j+1)\wedge \xi_i(x\wedge (-1)^j s) \wedge \eta_j(y\wedge t)\wedge u $. If we first increase $j$ and then increase $i$, we get first to $(i,j+1)\wedge x\wedge   \eta_j(y\wedge s)\wedge t\wedge u $ and then to $(i+1,j+1)\wedge \xi_i\big(x\wedge (-1)^{j+1} t\big) \wedge \eta_j(y\wedge s)\wedge u $. If I wanted to turn the first formula into the second I would have to substitute $s$ for $t$, $-t$ for $s$.

We conclude, then, that the family of subspaces we have considered is best described as 
\[X_i\wedge Y_j\wedge M(\tau) \wedge S^{n-i-j-2} \]
Here $M(\tau)$ is the Thom complex of a certain $2$-plane bundle $\tau$ over $\partial e$; more precisely, $\tau$ is obtained from $I\times \bR^2$ by identifying the two ends under the homeomoprhism $\begin{pmatrix}0&-1\\ 1&0 \end{pmatrix}$, So $\tau$ is an $\te{SO}(2)$-bundle over $\partial e=S^1$; it can be extended to a bundle over $e$. Of course there are different ways of extending $\tau$ to a bundle over $e$, since $\pi_1(\te{SO}(2))=\bZ$. But $\tau$ is essentially independent of $n$,$i$,$j$,$X$ and $Y$;
this follows from the description given above; or else one can use
coordinates to write down explicit isomorphisms which increase i by 1
 or j by 1. (The isomorphisms start from the identity map of $\bR^2$ over $[i]\times [j]$, and each suspension coordinate is either preserved or reversed
according to the demands of the signs.) All that is essential is that we
choose an extension of $\tau$ that is similarly independent of $n,i,j,X$ and $Y$. For example, with the description of $\tau$ given above, we can trivialize $\tau$ by using a geodesic path of Lent $\pi/2$ of $SO(2)$.

We take the part of $P_n$ corresponding of $e=e_{ij}$ to be \[X_i\wedge Y_j \wedge M(\tau) \wedge S^{n-i-j-2}\]
where $\tau$ now refers to the bundle as extended over $e_{ij}$. The identification with the part of $P_n$ already constructed is automatic. 

This completes the construction of $P_n=(X\wedge Y)_n$. The structure maps are obvious.

To summarize, we have constructed $(X\wedge Y)_n$ as a quotient space of \[\bigvee X_i\wedge Y_j \wedge M(\tau_d)\wedge S^{n-i-j-d}\]
Here the sum runs over cells $e_{ij}$ such that $i+j+\dim(e_{ij})\leq n$, and $d=\dim(e_{ij})$, and $\tau_d$ is a suitable$d$-plane bundle over $e_{ij}$. (For $d=0$ and $d=1,\tau_d$ was introduced as an explicitly trivialized bundle.) The identifications are obvious: we regard $X_i\wedge S^1$ as embedded in $X_{i+1}$,
\[X_i\wedge S^1 \wedge Y_j \wedge M(\tau) \wedge S^{n-i-j-d} \]
as
\[X_i\wedge Y_j\wedge M(1\oplus \tau) \wedge S^{n-i-j-d} \]

The discussion of the functoriality of $X\wedge Y$ goes exactly as for the telescope functor. More precisely, suppose $X'$ is cofinal in $X$, and we are given a function $f:X'\lar{}Z$ Then $X'\wedge Y$ is not cofinal in $X\wedge Y$, but we have the following functions.

\[\begin{tikzcd}
X\wedge Y && Z\wedge Y\\ X'\wedge Y \arrow[u,"i\wedge 1"] \arrow[urr, "f\wedge 1"]
\end{tikzcd}\]
When we pass to morphisms, $i'\wedge 1$ is an equivalence, by \ref{cor:p3ch03.5}, so we obtain a morphism from $X\wedge Y$ to $Z\wedge Y$. Since cylinders work right, we
conclude that this morphism depends only on the homotopy class of $f$. It is clear how one embeds $\te{Tel}(X\wedge_{BC} Y$ in $X\wedge Y$. The functions $\beta \alpha^{-1},\gamma \alpha^{-1}$ give a function \[\{0,1,2,3\dots\} \lar{}\{0,1,2,3\dots\}\times \{0,1,2,3\dots\} \]
In other words, they give the corners of a stepwise path in the quarter-plane $i\geq 0,j\geq 0$. We extend it to a function $\theta:\{k\geq 0\} \lar{} \{i\geq 0\}\times \{j\geq 0\}$ so that if $a\in B$,
\[\theta[\alpha(a),\alpha(a)+1]\subset [\beta(a),\beta(a)+1]\times [\gamma(a)] \]
and if $a\in C$,
\[ \theta[\alpha(a),\alpha(a)+1]\subset [\beta(a)]\times [\gamma(a),\gamma(a)] \]
The choice of $\theta$ is immaterial; two choices are homotopic through maps $\theta$ satisfying the same restrictions.

A typical part of $\te{Tel}(X\wedge_{BC} Y)$ is 
\[[k,k+1]^+ \wedge X_i\wedge Y_j \wedge S^1 \wedge S^{n-k-1}\]
where $i=\beta \alpha^{-1} k, j=\gamma \alpha^{-1} k$. We take the point \[t\wedge x\wedge y\wedge u\wedge v\]
and map it to 
\[\theta(t)\wedge x\wedge y\wedge u\wedge v\ \text{ in } \  e_{ij}^+\wedge X_i \wedge Y_j \wedge S^1 \wedge S^{n-i-j-1}\]
where $e_{ij}$ is the appropriate $1$-cell. Similarly for $[k]^+ \wedge X_i \wedge Y_j \wedge S^{n-k}$.

It is clear that changing the choice of $\theta$ only changes the resulting function \[\te{Tel}(X\wedge_{BC} Y) \lar{} X\wedge Y \]
by a homotopy. For any choice of $\theta$, the function $\te{Tel}(X\wedge_{BC}Y)\lar{}X\wedge Y$ is natural for functions of $X$ and $Y$ of degree $0$. From this, one has no difficulty in obtaining the naturality properties of $\te{eq}_{BC}$ in Theorem \ref{thm:p3ch04.2}.

We now prove Theorem \ref{thm:p3ch04.2}(iii). First we consider case $(a)$. So we suppose that $B$ and $C$ are infinite. We define a subspectrum $Q$ of $P$ as follows. Let $Q_{\alpha(a)}$ be the subcomplexes of $P_{\alpha(a)}$ corresponding to the cells $e_{ij}$ in the part of the quarter-plane given by $i'\leq \beta(a),j'\le \gamma(a)$. $Q_{\alpha(a)}$ admits a deformation retraction on $X_{\beta(a)}\wedge Y_{\gamma(a)}$, and $\te{Tel}(X\wedge_{BC} Y)_{\alpha(a)}$ admits a deformation retraction on $X_{\beta(a)}\wedge Y_{\gamma(a)}$. Hence, in the diagram
\[\begin{tikzcd}
& \te{Tel}(X\wedge_{BC}Y)\arrow[dr]&\\ X_{\beta(a)}\wedge Y_{\gamma(a)} \arrow[ur,"\cong"] \arrow[rr,"\cong"] && Q_{\alpha(a)}
\end{tikzcd}\]
the two inclusions marked induce isomorphisms of homotopy groups, so
the third one does also; passing to direct limits and applying \ref{cor:p3ch03.5}, the inclusion \[\te{Tel}(X\wedge_{BC}Y)\lar{}Q\]
is an equivalence.

It remains to consider cases $(b)$ and $(c)$, which are similar. Let us consider case $(b)$, so that $B$ is finite with $d$ members, and 
\[\xi_r:\Sigma X_r\lar{}X_{r+1}\] is an isomorphism for $r\geq d$. We now make a small change in the definition of $Q_{\alpha(a)}$ for "$a$" such that $\beta(a)\geq d$. For such $a$, we define $Q_{\alpha(a)}$ to be the subcomplex of $P_{\alpha(a)}$ corresponding to the cells $e_{ij}$ in the part of the quarter-plane given by $i'+j'\leq \alpha(a),j'\leq  \gamma(a)$. Then $Q_{\alpha(a)}$ still admits a deformation retraction onto $X_{\beta(a)}\wedge Y_{\gamma(a)}$, since the relevant map
\[X_d\wedge Y_j \wedge S^{n-d} \lar{} X_{d+e} \wedge Y_j \wedge S^{n-d-e}\quad (e\geq 0)\]
is an isomorphism. Also $Q$ is cofinal in $P$, so the proof carries over. We now turn to the proof of Theorem \ref{thm:p3ch04.5}.
\begin{lemma}\label{lem:p3ch04.6}
There is a spectrum $Q$ with homotopy equivalences $i_0:X\wedge Y\lar{} Q,i_1:Y\wedge X\lar{} Q$ so that the following diagram is commutative for each choice of $B$ and $C$ satisfying condition \ref{con:p3ch04.4}

\[\begin{tikzcd}
    & Q \\
    {X\wedge Y} && {Y\wedge X} \\
    {\text{Tel}(X\wedge_{BC}Y)} && {\text{Tel}(Y\wedge_{CB}X)} \\
    {X\wedge_{BC}Y} && {Y\wedge_{CB}X}
    \arrow["{c_{BC}}", from=4-1, to=4-3]
    \arrow[from=3-1, to=4-1]
    \arrow[from=3-1, to=2-1]
    \arrow[from=3-3, to=2-3]
    \arrow[from=3-3, to=4-3]
    \arrow["{i_0}", from=2-1, to=1-2]
    \arrow["{i_1}"', from=2-3, to=1-2]
\end{tikzcd}\]
\end{lemma}

This will certainly prove Theorem \ref{thm:p3ch04.5}; we have only to define $c$ to be $i_1^{-1}i_0$. Note that we do not have to discuss the naturality of $i_1^{-1} i_0$; it follows from that of the other morphisms in \ref{lem:p3ch04.6}.

To construct $Q$, we begin by taking a copy of $X\wedge Y$ and a copy of $Y\wedge X$. The remainder of the construction will be indexed ever the product
of the quarter-plane $i\geq 0,j\geq 0$ and the interval $I$. The endpoint $0$ of $I$ will correspond to $X\wedge Y$ and the endpoint $1$ of $I$ will correspond to $Y\wedge X$.

First we observe that we can make the following cells:
\begin{align*}
    ([i]\times[j]\times I)^+=([i]\times [j])^+\wedge I^+ && (\text{i or j even})\\
    ([i,i+1]\times [j]\times I)^+ && (\text{j even})\\
    ([i]\times [j,j+1]\times I)^+ && (\text{i even})
\end{align*}

The $n\nth$ term of the construction consists in taking the appropriate part of $(X\wedge Y)_n \wedge I^+$, identifying the end $0$ of the cylinder with the appropriate part of $(X\wedge Y)_n$, attaching the end $1$ of the cylinder to the appropriate part of $(Y\wedge X)$, by the following map: the point

\[(t,s)\wedge x\wedge y\wedge u\wedge v\text{  in  }e_{ij}^+ \wedge X_i\wedge Y_j \wedge S^d \wedge S^{n-i-j-d}\]
is to be identified with 
\[(s,t)\wedge x\wedge y\wedge u\wedge v\text{  in  }e_{ji}^+ \wedge Y_j\wedge X_i \wedge S^d \wedge S^{n-i-j-d}\]
These identifications are consistent.

Consider now a cell $e=[2i,2i+2]\times [2j,2j+2]\times I$. We have just described the part of $Q_n$ corresponding to the boundary $\partial e$ of $e$. Moreover, it contains a subcomplex of the following form:
\[X_{2i}\wedge Y_{2j}\wedge M(\tau') \wedge S^{n-2i-2j-4}.\]
Here $\tau'$ is a certain $4$-plane bundle over $\partial e$. This $4$-plane bundle depends only on the permutations and signs in our construction and on the extension $\tau$ chosen in the construction of $X\wedge Y$; it does not depend on the $n,i,j,X$ or $Y$. It is classified by an element \[\alpha\in \pi_1(\te{SO}) = \bZ_2\]

Suppose now that we consider the four cells, like the cell $e$ just considered, which make up the cell \[e'=[4i,4i+4]\times [4j+4j]\times I \]
Call them $e_1,e_2,e_3$ and $e_4$. The part of $Q$ so far constructed, corresponding to these cells, has a subcomplex of the form 
\[S^{n-4i-4j-8}\wedge M(\tau'')\wedge X_{4i}\wedge X_{4j} .\]
Here $M(\tau'')$ is the Thom complex of a certain $8$-plane bundle over \newline $\partial e_1\smile \partial e_2\smile \partial e_3\smile \partial e_4$. Over each $\partial e_i$ it restricts to the whitney sum of the previous bundle $\tau'$ and a trivial $4$-plane bundle. Therefore the restriction of $\tau''$ to $\partial e'$ is classified by $4\alpha=0$. Therefore $\tau''=0$ can be extended over $e'$.

From the previous construction, we now retain only $X\wedge Y, Y\wedge X,$ and the parts of the cylinder $(X\wedge Y)_n \wedge I^+$ with $i$ divisible by $4$ or $j$ divisible by $4$. We now add $X_{4i}\wedge Y_{4j} \wedge M(\tau'') \wedge S^{n-4i-4j-8}$ for each $i,j$, and $n$ such that $n\geq 4i+4j+8$. This completes the construction of $Q$.

The injections of $X\wedge Y$ and $Y\wedge X$ into $Q$ are clearly homotopy equivalences, by \ref{cor:p3ch03.5}. It is also clear that the diagram of Lemma \ref{lem:p3ch04.6} is commutative, because the relevant part of the cylinder $\te{Cyl}(X\wedge Y)$ was put in for that purpose.

This completes the proof of Lemma \ref{lem:p3ch04.6} and, therefore, of Theorem $\ref{thm:p3ch04.5}$.


We now turn to the proof of Theorem \ref{thm:p3c04.3}. The constructions $(X \wedge Y) \wedge Z$ and $X \wedge (Y \wedge Z)$ are ``quadruple telescopes'', indexed by a cell-decomposition of the positive cone in 4-space. We arrange to replace $(X \wedge Y) \wedge Z$ by an equivalent construction $P'$ and $X \wedge (Y \wedge Z)$ by an equivalent construction $P''$, so that both $P'$ and $P''$ are ``triple telescopes'', indexed by a cell-decomposition of the positive cone in 3-space. It will then be apparent that $P'$ and $P''$ are equivalent. More formally, we have the following lemma.

\begin{lemma}
\label{lem:p3c04.7}
There is a spectrum $P'$ and a homotopy equivalence $i' : P' \lar{} (X \wedge Y) \wedge Z$ (both independent of $B$, $C$ and $D$) such that the following diagram is commutative for each choice of $B$, $C$ and $D$. 
\[
\adjustbox{scale = 0.9, center} {
\begin{tikzcd}
(X \wedge_{BC} Y) \wedge Z                                                                                                  &  & (\mathrm{Tel}(X \wedge_{BC} Y)) \wedge Z \arrow[r, "i \wedge 1"] \arrow[ll, "r \wedge 1"'] & (X \wedge Y) \wedge Z \\
{\mathrm{Tel}((X \wedge_{BC} Y) \wedge_{B \cup C, D} Z)} \arrow[d, "r"] \arrow[u, "i"] \arrow[rrr, "k'"] \arrow[rru, "j'"'] &  &                                                                                                     & P' \arrow[u, "i'"']   \\
{(X \wedge_{BC} Y) \wedge_{B \cup C, D} Z}                                                                                  &  &                                                                                                     &                      
\end{tikzcd}
}\]

Similarly for $X \wedge (Y \wedge Z)$ with $i'$, $j'$, $k'$ and $P'$ replaced by $i''$, $j''$, $k''$ and $P''$. Moreover, there is a homotopy equivalence $P' \lar{e} P''$ such that the following diagram is commutative. 

\begin{center}
\begin{tikzcd}
P' \arrow[rr, "e"]                                                                                                &  & P''                                                                                 \\
{\mathrm{Tel}((X \wedge_{BC} Y) \wedge_{B \cup C, D} Z)} \arrow[u, "k'"] \arrow[rr, equals] &  & {\mathrm{Tel}(X \wedge_{B, C \cup D} (Y \wedge_{\mathrm{CD}} Z))} \arrow[u, "k''"']
\end{tikzcd}
\end{center}
\end{lemma}

\begin{proof}
By definition, the $n^{\mathrm{th}}$ term of $(X \wedge Y) \wedge Z$ is a union \[\bigcup_{hk} (X \wedge Y)_h \wedge Z_k \wedge M(\tau_\delta) \wedge S^{n - h - k - \delta}\] where the union extends over cells $e_{hk}$ such that \[h + k + \dim e_{hk} \le n,\] $\delta = \dim e_{hk}$ and $\tau_\delta$ is a $\delta$-plane bundle. That is, it is a union \[\bigcup_{e_{ij}, e_{hk}} X_i \wedge Y_j \wedge M(\tau_d) \wedge S^{h - i - j - d} \wedge Z_k \wedge M(\tau_\delta) \wedge S^{n - h - k - \delta}\] where $e_{ij}$ runs over cells with \[i + j + \dim e_{ij} \le h,\] $d = \dim(e_{ij})$ and $\tau_d$ is a $d$-plane bundle. We arrange this as \[\bigcup_{e_{ik}, e_{hk}} X_i \wedge Y_j \wedge Z_k \wedge M(\tau_d \oplus \tau_\delta) \wedge S^{h - i - j - d} \wedge S^{n - h - k - \delta}.\] Thus the construction is indexed over a cell-decomposition of the positive cone $i \ge 0$, $j \ge 0$, $h \ge 0$, $k \ge 0$ in $4$-space. Call this cone $C^4$. Let $C^3$ be the positive cone $i \ge 0$, $j \ge 0$, $k \ge 0$ in $3$-space, and divide $C^3$ into cells in the obvious way, so that the cells are $r$-cubes of side $1$ for $r = 0, 1, 2, 3$. 

We construct $P'$ by giving a suitable cellular map $\theta$ from $C^3$ to $C^4$ by ``pulling back'' the bundles and complexes we have associated with the parts of $C^4$. Actually we construct $\theta$ to preserve the $k$-coordinate, so it is only necessary to construct a map $\varphi$ from the positive cone $i \ge 0$, $j \ge 0$ to the positive cone $i \ge 0$, $j \ge 0$, $h \ge 0$.

Our idea in defining $\theta$ and $\varphi$ is to only use the cells $e_{ij} \times e_{hk}$ such that \[i + j + \dim e_{ij} = h;\] firstly because the other parts of $(X \wedge Y) \wedge Z$ are redundant, and secondly because by keeping $S^{h - i - j - d} = S^0$ we avoid suspension coordinates in the wrong place.

 We first indicate into which subcomplexes the cells are to be mapped 

\begin{align*}
\varphi([i] \times [j]) & = [i] \times [j] \times [i + j] \\ \phi([i, i + 1] \times [j]) & \subset ([i] \times [j] \times [i + j, i + j + 1]) \\ & \cup ([i, i + 1] \times [j] \times [i + j + 1]) \\ \phi([i] \times [j, j + 1]) & \subset ([i] \times [j] \times [i + j + 1]) \\ & \cup ([i] \times [j, j + 1] \times [i + j + 1]) \\ \phi([i, i + 1] \times [j, j + 1]) & \subset ([i] \times [j] \times [i + j, i + j + 1]) \\ & \cup ([i, i + 1] \cup [j] \times [i + j + 1, i + j + 2]) \\ & \cup ([i] \times [j, j + 1] \times [i +j + 1, i + j + 2]) \\ & \cup ([i, i + 1] \times [j, j + 1] \times [i + j + 2]).
\end{align*}

In each case the proposed subcomplex is contractible, so the construction of $\phi$ is possible and unique up to homotopy. In each case, the image of $\phi$ must be the whole subcomplex given, so we can refer to the subcomplex as $\phi(e_{ij})$. Similarly for $\theta(e_{ijk})$.

We next note that for each cell $e_{ijk}$ such that $i + j + k + \dim(e_{ijk}) \le n$, the part of $((X \wedge Y) \wedge Z)_n$ associated with $\theta(e_{ijk})$ contains a subcomplex of the form \[X_i \wedge Y_j \wedge Z_k \wedge M(\tau_d) \wedge S^{n - i - j - k - d},\] where $d = \dim(e_{ijk})$ and $\tau_d$ is a $d$-plane bundle over $\theta(e_{ijk})$. We decompose the corresponding part of $P'$ to be \[X_i \wedge Y_j \wedge Z_k \wedge M(\theta^\ast \tau_d) \wedge S^{n - i - j - k - d},\] where $\theta^\ast \tau_d$ is the induced bundle over $e_{ijk}$. The map $i'$ on this part of $P'$ is induced by the map of bundles $\theta^\ast \tau_d \lar{} \tau_d$ over the map $\theta$ of spaces. The identifications to be made assembling $P'$ are automatic; one just pulls back the identification of $(X \wedge Y) \wedge Z$. 

We make the structure of $P'$ more explicit. Corresponding to the 0-cells $e_{ijk}$ we have \[\bigvee_{i + j + k \le n} X_i \wedge Y_j \wedge Z_k \wedge ([i] \times [j] \times [k])^+ \wedge S^{n - i - j - k}.\] Corresponding to 1-cells we have \[\bigvee_{i + j + k + 1 \le n} e_{ijk}^+ \wedge X_i \wedge Y_j \wedge Z_k \wedge S^1 \wedge S^{n - i - j - k - 1}.\] Here the attaching maps are the obvious ones, involving the obvious signs. 

For each 2-cell $e = e_{ijk}$, the bundle $\theta^\ast \tau$ over $e$ is exactly as described in the construction of $X \wedge Y$.

For each 3-cell $e = e_{ijk}$, there is only one bundle over $e$ extending the given bundle $\theta^\ast \tau$ over $\partial e$, since $\pi_3(\mathrm{BSO}(3)) = \pi_2(\mathrm{SO}(3)) = 0$. So we need not worry which bundle arises. 

On the other hand, the description of $P''$ is exactly the same as the description we have just given for $P'$. This provides the map $e : P' \lar{} P''$. 

The map \[k' : \mathrm{Tel}((X \wedge_{BC} Y) \wedge_{B \cup C, D} Z) \lar{} P'\] is basically obvious. The functions $\beta \alpha^{-1}$, $\gamma \alpha^{-1}$ and $\delta \alpha^{-1}$ give a function $\theta' : \{0, 1, 2, 3, \ldots\} \to \{0, 1, 2, 3, \ldots\}^3$. We extend it to a function $\theta''$ mapping each cell of $C^1$ (the positive half-line with our usual cell structure) into the obvious cell of $C^3$. Now we construct the map $k'$ as we constructed the map $\mathrm{Tel}(X \wedge_{BC} Y) \lar{} X \wedge Y$.
%CORRECTION: fixed missing parentheses from source text after "this defines the function..."

We now observe that the function \[i'k' : \mathrm{Tel}((X \wedge_{BC} Y) \wedge_{B \cup C, D} Z) \lar{} (X \times Y) \wedge Z\] actually maps into $\left\{\mathrm{Tel}(X \wedge_{BC} Y)\right\} \wedge Z$; this defines the function \[j' : (\mathrm{Tel}(X \wedge_{BC} Y)) \wedge_{B \cup C, D} Z \lar{} (\mathrm{Tel}(X \wedge_{BC} Y)) \wedge Z.\] The function $(r \wedge 1) j'$ satisfies the definition for $i$. (Some of the cylinders spend some of their time stationary and the rest hurrying to make up for it, but this is allowed) 

This completes the proof of Lemma~\ref{lem:p3c04.7}, which completes the proof of \ref{thm:p3c04.3}, which completes the proof of Theorem~\ref{thm:p3c04.1} so far as it refers to maps of degree $0$. 
\end{proof}

We now propose to go back and recover the properties of our constructions with respect to maps of non-zero degree.

First we introduce the sphere-spectra of different stable dimensions. Let us define the spectrum $\underline S^i$ by \[(\underline S^i)_n = \begin{cases}S^{n + 1} & n + i \ge 0 \\ \mathrm{pt.} & n + i < 0.\end{cases}\]

\begin{proposition}
We have an equivalence $\underline S^i \wedge \underline S^j \lar{e} \underline{S}^{i + j}$ such that the following diagrams are commutative.

\begin{center}
\begin{tikzcd}
(\underline S^i \wedge \underline S^j) \wedge \underline S^k \arrow[rr, "a"] \arrow[d, "e \wedge 1"] &                          & \underline S^i \wedge (\underline S^j \wedge \underline S^k) \arrow[d, "1 \wedge e"] \\
\underline S^{i + j} \wedge \underline S^k \arrow[rd, "e"']                                          &                          & \underline S^i \wedge \underline S^{j + k} \arrow[ld, "e"]                           \\
                                                                                                     & \underline S^{i + j + k} &                                                                                     
\end{tikzcd}
\end{center}

\begin{center}
\begin{tikzcd}
\underline S^i \wedge \underline S^j \arrow[d, "e"'] \arrow[rr, "c"] &  & \underline S^j \wedge \underline S^i \arrow[d, "e"] \\
\underline S^{i + j} \arrow[rr, "(-1)^{ij}"]                         &  & \underline S^{j + i}                               
\end{tikzcd}
\end{center}
%TODO: not quite nailed these two
\[
\begin{tikzcd}[every arrow/.append style={shift left}]
 \underline S^0 \wedge \underline S^j \arrow[bend left]{r}{e} \arrow[bend right]{r}{\ell} & \underline S^j
\end{tikzcd}
\]

\[
\begin{tikzcd}[every arrow/.append style={shift left}]
 \underline S^i \wedge \underline S^0 \arrow[bend left]{r}{e} \arrow[bend right]{r}{r} & \underline S^i
\end{tikzcd}
\]

\end{proposition}

\begin{proof}
\begin{enumerate}
    \item[(i)] Any handicrafted smash-product of $\underline S^i$ and $\underline S^j$ gives a spectrum that has the same terms as $\underline S^{i + j}$ from some point onwards. We just take care to pick an equivalence that is orientation-preserving.
    \item[(ii)] $[\underline S^i, \underline S^j] = \lim_{n \to \infty} [S^{n + i}, S^{n + i}] = \mathbb Z$; so to check the commutativity of any such diagram, we have only to check the degree of a map. We have been careful to make all our constructions so as to do the right thing on orientations.
\end{enumerate}
\end{proof}

\begin{proposition}
\label{prop:p3c04.9}
We have the equivalences \[\gamma_r : X \lar{} \underline S^r \wedge X \qquad \text {(of degree } r\text{)}\] with the following properties. 

\begin{enumerate}
    \item[(i)] $\gamma_r$ is natural for maps of $X$ of degree $0$. (This is all we can ask, because we have not yet made $\underline S^r \wedge X$ functorial for maps of non-zero degree.),
    \item[(ii)] $\gamma_0 = \ell^{-1}$,
    \item[(iii)] The following diagram is commutative for each $r$ and $s$.
    \begin{center}
    \begin{tikzcd}
\underline S^{r + s} \wedge X                          &  & (\underline S^r \wedge \underline S^s) \wedge X \arrow[d, "a"] \arrow[ll, "e \wedge 1"'] \\
                                                       &  & \underline S^r \wedge (\underline S^s \wedge X)                                          \\
X \arrow[uu, "\gamma_{r + s}"'] \arrow[rr, "\gamma_s"] &  & \underline S^s \wedge X \arrow[u, "\gamma_r"']                                          
\end{tikzcd}
    \end{center}
\end{enumerate}
\end{proposition}

\begin{proof}
Clearly if we take $\gamma_0 = \ell^{-1}$, it is natural for maps of $X$ of degree $0$. Consider now \[\underline S^0 \wedge_{1, \{2, 3, \ldots\}} X \quad \text { and} \quad \underline S^1 \wedge_{\emptyset, \{1, 2, 3, \ldots\}} X.\] On the left, the $n^{\mathrm{th}}$ term is $S^1 \wedge X_{n - 1}$; on the right, the $(n - 1)$-st term is $S^1 \wedge X_{n - 1}$. The structure maps are the same in both cases. So the identity maps $S^1 \wedge X_{n - 1} \lar{} S^1 \wedge X_{n - 1}$ give the components of an equivalence of degree $+1$ \[\underline S^0 \wedge X \lar{} \underline S^1 \wedge X.\] It is clearly natural for maps of $X$ of degree $0$. Composing with $\ell^{-1}$, we obtain an equivalence $\gamma_1$. 

Note that at this point I have essentially picked up the Puppe Desuspension Theorem, without the restrictive hypotheses.

Now I define $\gamma_s$ for all other values of $s$ by induction upwards and downwards over $s$, making the following diagram commutative. 

\begin{center}
\begin{tikzcd}
\underline S^{1 + s} \wedge X                          &  & (\underline S^1 \wedge \underline S^s) \wedge X \arrow[d, "a"] \arrow[ll, "e \wedge 1"'] \\
                                                       &  & \underline S^1 \wedge (\underline S^s \wedge X)                                          \\
X \arrow[uu, "\gamma_{1 + s}"'] \arrow[rr, "\gamma_s"] &  & \underline S^s \wedge X \arrow[u, "\gamma_1"']                                          
\end{tikzcd}
\end{center}

One has to check that this is consistent for $s = 0$. Note also that $\gamma_{1 + s}$ or $\gamma_s$, whichever is being defined, is natural for maps of $X$ of degree $0$, because all the ingredients of its definition are so. 

We now prove the commutativity of the diagram

\begin{center}
\begin{tikzcd}
S^{r + s} \wedge X                                     &  & (\underline S^r \wedge \underline S^s) \wedge X \arrow[ll, "e \wedge 1"'] \arrow[d, "a"] \\
                                                       &  & \underline S^r \wedge (\underline S^s \wedge X)                                          \\
X \arrow[rr, "\gamma_s"] \arrow[uu, "\gamma_{r + s}"'] &  & \underline S^s \wedge X \arrow[u, "\gamma_r"']                                          
\end{tikzcd}
\end{center}

by induction upwards and downwards over $r$. Here we start from the cases $r = 0$ (which is a trivial verification) and $r = 1$ (which holds by the construction of $\gamma_s$). The inductive step is diagram-chasing.

We are now ready to replace our original graded category by one which appears slightly different. In the new category, the objects are CW-spectra just as before; but the morphisms of degree $r$ are given by \[[\underline S^r \wedge X, Y]_0\] in the old category. Composition is done as follows. Suppose given \[\underline S^r \wedge X \lar{f} Y, \quad \underline S^s \wedge Y \lar{f} Z\] of degree 0; take their composite to be \[S^{s + r} \wedge X \xleftarrow{e \wedge 1} (\underline S^s \wedge \underline S^r) \wedge X \lar{a} \underline S^s \wedge (\underline S^r \wedge X) \vra{1 \wedge f} \underline S^s \wedge Y \lar{g} Z.\] One has to check that composition is associative, and that $\ell : \underline S^0 \wedge X \lar{} X$ is an identity map. This is easy.
\end{proof}

\begin{proposition}\label{prop:p3ch04.10}
The new graded category is isomorphic to the old, under the isomorphism sending \[\underline S^r \wedge X \lar{f} Y \quad \text {(in the new category)}\] to \[X \lar{\gamma_r} \underline S^r \wedge X \lar{f} Y \quad \text{(in the old category).}\]
\end{proposition}

\begin{proof}
Since $\gamma_r$ is an equivalence in the old category, it is clear that this gives a 1-1 correspondence between $[\underline S^r \wedge X, Y]_0$ (that is, the set of morphisms of degree $r$ in the old category). It remains only to check that this one-to-one correspondence preserves composition and identity maps. But this is immediate from the properties of $\gamma_r$ in Proposition~\ref{prop:p3c04.9}.
\end{proof}

If you want to see what you are doing with maps of degree $r$, I really recommend considering them as maps $\underline S^r \wedge X \lar{} Y$ of degree $0$. In particular, it is easy to see how to make $X \wedge Y$ functorial on the new category. More precisely, suppose given morphisms in the new category \[S^r \wedge X \lar{f} X', \qquad \underline S^s \wedge Y \lar{g} Y'.\] Then we define their smash-product to be \[\underline S^{r + s} \wedge X \wedge Y \xleftarrow{e \wedge 1 \wedge 1} \underline S^r \wedge S^s \wedge X \wedge Y \lar{1 \wedge c \wedge 1} \underline S^r \wedge X \wedge \underline S^s \wedge Y \lar{f \wedge g} X' \wedge Y'.\]

To prove that this has all the properties mentioned in Theorem~\ref{thm:p3c04.1} is now a routine exercise of diagram-chasing. At the same time, we check that we have not altered the definition of $f \wedge g$ if $f$ and $g$ happen to be of degree $0$. 


This completes the proof of Theorem~\ref{thm:p3c04.1}.

\begin{exercise}
Show that the naturality of $\gamma_r$ with respect to maps of degree $s$ is as follows: the diagram 

\begin{center}
\begin{tikzcd}
X \arrow[rr, "\gamma_r"] \arrow[dd, "f"'] &           & \underline S^r \wedge X \arrow[dd, "1 \wedge f"] \\
                                          & (-1)^{rs} &                                                  \\
Y \arrow[rr, "\gamma_r"]                  &           & \underline S^r \wedge Y                         
\end{tikzcd}
\end{center}

is commutative up to a sign of $(-1)^{rs}$ if $f \in [X, Y]_s$.
\end{exercise}

\begin{proposition}\label{prop:p3ch04.11}
The smash-product is distributive over the wedge-sum. Let $X = \bigvee_\alpha X_\alpha$; let $i_\alpha : X_\alpha \lar{} X$ be a typical inclusion. Then the morphism \[\bigvee_\alpha (X_\alpha \wedge Y) \vra{\{i_\alpha \wedge 1\}} \bigg(\bigvee_\alpha X_\alpha\bigg) \wedge Y\] is an equivalence. 
\end{proposition}

\begin{proof}
Use a suitable handicrafted smash-product. 
\end{proof}

\begin{proposition}\label{prop:p3ch04.12}
Let $X \lar{f} Y \lar{i} Z$ be a cofibering (it is sufficient to consider morphisms of degree zero). Then \[W \wedge X \vra{1 \wedge f} W \times Y \vra{1 \wedge i} W \wedge Z\] is also a cofibering.
\end{proposition}

\begin{proof}
It suffices to check for the case in which $f : X \lar{} Y$ is the inclusion of a closed subspectrum, $i : Y \lar{} Z$ is the projection $Y \lar {} Y/X$ and $\bigwedge = \bigwedge_{BC}$.
\end{proof}
\end{document}