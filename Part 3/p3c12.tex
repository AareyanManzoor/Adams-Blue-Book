\documentclass[../main]{subfiles}

\renewcommand{\labelenumi}{(\roman{enumi})}

\begin{document}

\chapter{The Steenrod Algebra and its Dual}
\label{sec:p3c12}
One knows that in order to perform calculations in ordinary cohomology, it is very useful to have operations like Steenrod squares.

In the general case, let $E$ be a spectrum. Then to every element of $E^\ast(E)$ we can associate a natural transformation $E^\ast(X)\longrightarrow E^\ast(X)$ defined for all spectra $X$. Namely, given
\[ X\overset f\longrightarrow E \text{ and } E\overset g\longrightarrow E, \]
we form $X\overset{gf}\longrightarrow E$. This gives a $1-1$ correspondence between elements of $E^\ast(E)$ and such natural transformations (consider the case $X=E$).

Now $E^\ast(E)$ is of course a group; addition in it corresponds to adding operations
\[ (g_1+g_2)f = (g_1f) + (g_2f). \]
But $E^\ast(E)$ is in fact a ring; multiplication in it % its (sic)
corresponds to composing operations,
\[ (g_1g_2)f = g_1(g_2f) \]

\begin{examples}

Suppose given a prime $p$; take $E=H\bZ_p$. Then $A^\ast={(H\bZ_p)}^\ast(H\bZ_p)$ is the \emph{mod $p$ Steenrod algebra}\index{Steenrod algebra mod p}, the algebra of stable cohomology operations on ordinary chomology with $\bZ_p$ coefficients. That it is an algebra over $\bZ_p$ is clear from the fact that it contains $\bZ_p$.

\end{examples} % TODO: Not sure where the example is supposed to end...

It is a fact that $A^\ast$ is generated by the Steenrod operations. If $p=2$ these are Steenrod squares
\[ \mathrm{Sq}^i\colon H^n(X, Y; \bZ_2) \longrightarrow H^{n+i}(X, Y; \bZ_2) \]

If $p>2$ these are the Steenrod powers
\[ P^k\colon H^n(X, Y; \bZ_p) \longrightarrow H^{n+2k(p-1)}(X, Y;\bZ_p) \]
together with the Bockstein boundary
\[ \beta_p\colon H^n(X,Y;\bZ_p)\longrightarrow H^{n+1}(X,Y;\bZ_p) \]
The fact that $A^\ast$ is generated by the Steenrod operations is not obvious, and should not be taken as a definition; it comes from the calculation of ${(H\bZ_p)}^\ast(H\bZ_p)$ which is due to Serre for $p=2$, and to Cartan for $p>2$.

Actually $A^\ast$ has more structure than just the structure of an algebra. Before going into this, I want to comment on the work of Milnor \cite{milnor}. Milnor showed that it is also good to look at the dual of the Steenrod algebra,
\[ A_\ast=(H\bZ_p)_\ast(H\bZ_p). \]
Here $A_\ast$ and $A^\ast$ are dual graded vector spaces over $\bZ_p$. Of course, if we did not know that $A_\ast$ is finite-dimensional over $\bZ_p$ in each degree we would only say
\[ A^n=\Hom_{\bZ_p}(A_n,\bZ_p); \]
but of course we do know it.

Now $H\bZ_p$ is a ring-spectrum; we have a map
\[ \mu\colon H\bZ_p\wedge H\bZ_p\longrightarrow H\bZ_p \]
So we get
\begin{align*}
    A_\ast\otimes A_\ast &= (H\bZ_p)_\ast(H\bZ_p)\otimes(H\bZ_p)_\ast(H\bZ_p) \\
    &\overset{\underline{\wedge}}\longrightarrow(H\bZ_p)_\ast(H\bZ_p\wedge H\bZ_p) \\
    &\overset{\mu_\ast}\longrightarrow(H\bZ_p)_\ast(H\bZ_p)=A_\ast.
\end{align*}
So $A_\ast$ also is an algebra.

The dual of the product map $\phi\colon A_\ast\otimes A_\ast\longrightarrow A_\ast$ is of course a coproduct map $\psi=\phi^\ast\colon A^\ast\longrightarrow A^\ast\otimes A^\ast$. The interpretation of this coproduct is as follows. Suppose
\[ \psi(a) = \sum_ia'_i\otimes a''_i \]
Then
\[ a(x\barwedge y)=\sum_i(-1)^{\abs{a''_i}\abs{x}}(a'_ix)\barwedge(a''_iy) \quad \text{(Cartan formula).} \]
There exists one and only one element $\sum\limits_i a'_i\otimes a''_i$ such that this formula is true for all $x$ and $y$. Of course the formula is then true for $x\bar\times y$ % TODO: \bar or \overline?
and $x\smile y$. % TODO: \smile or \cup
For example,
\[ \mathrm{Sq}^k(xy)=\sum_{i+j=k} (\mathrm{Sq}^ix)(\mathrm{Sq}^iy) \]
so that
\[ \psi \mathrm{Sq}^k=\sum_{i+j=k} \mathrm{Sq}^i\otimes \mathrm{Sq}^j. \]

It can easily be shown that in this way $A^\ast$ becomes a Hopf algebra. Dually, $A_\ast$ becomes a Hopf coalgebra; its coproduct is the dual of the composition product in $A^\ast$. %to check, is it hopf coalgebra?

More generally, let $X$ be a space such that $(H\bZ_p)_\ast(X)$ is finite-dimensional in each degree. Then $(H\bZ_p)^\ast(X)$ is a module over $A^\ast$. The action is given by a map:
\[ A^\ast\otimes{(H\bZ_p)}^\ast(X)\longrightarrow{(H\bZ_p)}^\ast(X) . \]
The dual of this map is a coaction map:
\[ (H\bZ_p)_\ast(X)\longrightarrow A_\ast\otimes(H\bZ_p)_\ast(X) . \]
Thus $(H\bZ_p)_\ast(X)$ becomes a comodule over the coalgebra $A_\ast$. The assumption that $(H\bZ_p)_\ast(X)$ is locally finite-dimensional is in fact unnecessary, since the coaction map can be defined directly, as will be done below in a more general setting.

It turns out that the structure of $A_\ast$ is very much easier to describe than the structure of $A^\ast$. One reason is that the product in $A_\ast$ is commutative, whereas that in $A^\ast$ is not ($\mathrm{Sq}^1\mathrm{Sq}^2\neq \mathrm{Sq}^2\mathrm{Sq}^1$).

We give a description for the case $p=2$. We start from $\mbb{RP}^\infty$, which is an Eilenberg-MacLane space of type $(\bZ_2,1)$. We have ${(H\bZ_2)}^\ast(\mbb{RP}^\infty)=\bZ_2[x]$, a polynomial algebra on one generator $x$ of dimension $1$ (the fundamental class). We may take in $(H\bZ_2)_\ast(\mbb{RP}^\infty)$ a base of elements $b_i\in{(H\bZ_2)}_i(\mbb{RP}^\infty)$ such that
\[ \iprod{x^i, b_j} = \delta_{ij} . \]
Since $\mbb{RP}^\infty$ is term $1$ in the $H\bZ_2$ spectrum, $b_j$ yields some element in \newline ${(H\bZ_2)}_{j-1}(H\bZ_2)=A_{j-1}$. It can easily be shown that this element is zero unless $j$ is a power of $2$. We define $\xi_n$ to be the image of $b_{2^n}$ in $A_{2^n-1}$. The element $\xi_0$ turns out to be the unit $1\in A_0$.

\begin{theorem}[Serre-Milnor] \label{thm:p3ch12.1}
If $p=2$,
\[ A_\ast=\bZ_2[\xi_1,\xi_2,\dots] . \]
\end{theorem}

The proof is non-trivial, and is omitted here.

The construction of $\xi_i$ yields the following description of $\xi_i$ as a linear function on $A^\ast$.

\begin{proposition} \label{prop:p3ch12.2}
The action of $a\in A^\ast$ on ${(H\bZ_2)}^1(\mbb{RP}^\infty)$ is given by
\[ ax = \sum_{i\geq0}\iprod{a,\xi_i}x^{2^i} . \]

\end{proposition}

For $x$ is a morphism and the suspension spectrum of $\mbb{RP}^\infty$ to $H\bZ_2$ of degree $-1$, and
\[ \iprod{ax,b_j} = \iprod{x^\ast a,b_j} = \iprod{a,x_\ast b_j} = \begin{cases} 0 & \text{if $j\neq 2^r$ for some $r$} \\ \iprod{a,\xi_r} & \text{if } j=2^r \end{cases} . \]

From this it is rather easy to work out the effect of $a$ on $x^2$, $x^4$, etc. We get:

\begin{proposition} \label{prop:p3ch12.3}
\[ a\Big(x^{2^i}\Big) = \sum_{j\geq0}\iprod{a,\xi_j^{2^i}}x^{2^{i+j}} \]
\end{proposition}

It now becomes easy to work out the effect of a composite $ba$ on $x$, which gives us $\iprod{ba,\xi_i}$ and therefore $\psi\xi_i$.

\begin{proposition} \label{prop:p3ch12.4}
\[ \psi\xi_k = \sum_{i+j=k}\xi_j^{2^i}\otimes\xi_i \]
\end{proposition}

We would now like to carry over some of this work to generalised homology theories. Let $E$ be a ring-spectrum with multiplication $\mu$. Then obviously the appropriate generalisation of $A_\ast$ is $E_\ast(E)$. It turns out that this works quite well even in various cases where $E^\ast(E)$ works horribly badly. However, one needs an assumption and one must give a warning. The warning is that in the classical case $A_\ast$ is an algebra over $\bZ_p$, but in the generalised case $E_\ast(E)$ is a bimodule over $\pi_\ast(E)$. There are two actions of $\pi_\ast(E)$ on $E_\ast(E)$, and one has to remember that they are different. The left action $\pi_\ast(E)\otimes E_\ast(E)\longrightarrow E_\ast(E)$ is obtained by using the morphism $E\wedge E\wedge E\overset{\mu\wedge1}\longrightarrow E\wedge E$; the right action $E_\ast(E)\otimes\pi_\ast(E)\longrightarrow E_\ast(E)$ is obtained by using the morphism $E\wedge E\wedge E\overset{1\wedge\mu}\longrightarrow E\wedge E$.

The assumption we have to make is that $E_\ast(E)$ is flat as a right module over $\pi_\ast(E)$. I say ``as a right module'', but if $E$ is commutative, which is the usual case it is equivalent to say that $E_\ast(E)$ is flat as a left module; this is seen by using $c\colon E\wedge E\longrightarrow E\wedge E$ to interchange the two sides.

The assumption is satisfied for the following cases: \newline $E=\te{KO, K, MO, MU, MSp}$, $S$, and $H\bZ_p$. % TODO: should these be blackboard bold?
See \cite{adams3}, Lemma 28, p. 45.

With this assumption, we have the following lemma. Consider the morphism
\[ (E\wedge E)\wedge (E\wedge X)\xrightarrow{1\wedge\mu\wedge1} E\wedge E\wedge X . \]

It induces a product map
\[ E_\ast(E)\otimes_{\pi_\ast(E)}E_\ast(X)\longrightarrow[S,E\wedge E\wedge X]_\ast \]

\begin{lemma} \label{lem:p3ch12.5}
The product map is an isomorphism.
\end{lemma}

\begin{proof}

\begin{enumerate}
    \item If $X=S^P$, the result is trivial.
    \item If we have a cofibering:
    \[ X_1\longrightarrow X_2\longrightarrow X_3\longrightarrow X_4\longrightarrow X_5 \]
    and the result is true for $X_1$, $X_2$, $X_4$, and $X_5$, then it is true for $X_3$ (by the $5$-lemma).
    \item The result is true if $X$ is any finite spectrum, by induction on the number of cells, using (i) and (ii).
    \item The result is true if $X$ is any spectrum, by passing to direct limits.
\end{enumerate}

\end{proof}

We can now define the coaction map we want. Consider the morphism
\[ E\wedge X\simeq E\wedge S\wedge X\xrightarrow{1\wedge i\wedge 1}E\wedge E\wedge X . \]
This induces
\[ E_\ast(X)\xrightarrow{(1\wedge i\wedge 1)_\ast}[S,E\wedge E\wedge X]_\ast . \]
Composing this with the inverse of the isomorphism in Lemma \ref{lem:p3ch12.5}, we obtain a homomorphism
\[ \psi_X\colon E_\ast(X)\longrightarrow E_\ast(E)\otimes_{\pi_\ast(E)}E_\ast(X) . \]
Specialising to the case $X=E$, we obtain the homomorphism:
\[ \psi_E\colon E_\ast(X)\longrightarrow E_\ast(E)\otimes_{\pi_\ast(E)}E_\ast(E) . \]

We also define a counit map
\[ \varepsilon\colon E_\ast(E)\longrightarrow\pi_\ast(E) , \]
which is simply the homomorphism induced by the product morphism
\[ \mu\colon E\wedge E\longrightarrow E . \]

\begin{theorem} \label{thm:p3ch12.6}

\begin{enumerate}
    \item $E_\ast(E)$ is a coalgebra with $\psi_E$ as a coproduct map and $\varepsilon$ as a counit map.
    \item $E_\ast(X)$ is a comodule over $E_\ast(E)$ with $\psi_X$ as the coaction map.
    \item If $E=H\bZ_p$, then $\psi_X$, $\psi_E$, and $\varepsilon$ become the structure maps classically considered.
\end{enumerate}

To give a complete proof of \ref{thm:p3ch12.6}, one has to introduce a few more structure maps, which is very easy, and check their properties by diagram chasing. See \cite{adams3} chapter 3.

\end{theorem}

\end{document}