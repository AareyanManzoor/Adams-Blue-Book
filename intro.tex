\documentclass[main]{subfiles}
\begin{document}

\chapter*{Preface to the original}
The three sections of this book represent courses of lectures which I delivered at the University of Chicago in 1967, 1970 and 1971 respectively; and the three sections are of slightly different characters.
The 1967 lectures dealt with part of Novikov's work on complex cobordism while that work was still new---they were prepared before I had access to a translation of Novikov's full-length paper, \cite{novikov3}. 
They were delivered as seminars to an audience assumed to be familiar with algebraic topology. 
The 1970 lectures also assumed familiarity, but were a longer series attempting a more complete exposition; I aimed to cover Quillen's work on formal groups and complex cobordism.
Finally, the 1971 lectures were a full-length ten-week course, aiming to begin at the beginning and cover many of the things a graduate student needs to know in the area of stable homotopy and generalised homology theories.
They form two-thirds of the present book.

No attempt has been made to rewrite the three sections to impose uniformity, whether of notation or of anything else. 
Each section has its own introduction, where the reader may find more details of the topics considered. 
Each section has its own system of references; in Part~\ref{part:p1} the
references are given where they are needed; in Part \ref{part:p2} the references are collected at the end, with Part \ref{part:p1} as a reference; in Part \ref{part:p1} the references are again at the end, with Part \ref{part:p1} as a reference. 
However, the page numbers given in references to Part \ref{part:p1} refer---I hope---to pages in the present book.

Although I have not tried to impose uniformity by rewriting, a certain unity of theme is present. 
Among the notions with which familiarity is assumed near the beginning of Part~\ref{part:p1}, I note the following: spectra, products, and the derived functor of the inverse limit. 
All these matters are treated in Part~\ref{part:p3}-~in sections 2--3, 9 and 8. Similarly, near the beginning of Part~\ref{part:p2}, I assume it known that a spectrum determines a generalised homology theory and a generalised cohomology theory; this is set out in Part 3, section \ref{sec:p3c06} . 
Again, at the end of Part 1, section \ref{sec:p1c2}  the reader is referred to the literature for information on $\pi_{*} (\te{MU})$; they could equally well go to Part 2, section \ref{sec:p2c6}. Perhaps one should infer that in my choice of material, methods and results for my later courses, I was influenced by the applications I had already lectured on, as well as others I knew.

I am conscious of other places where the three parts of this book overlap, but perhaps the reader can profit by analysing these overlaps for themself; and certainly they should feel free to read the parts in an order reflecting his own taste. 
I need hardly direct the expert; a newcomer to the subject would probably do best to begin by taking what they need from the first ten sections of Part~\ref{part:p3}.

I would like to express my thanks to my hosts in the University of Chicago, and to R.~Ming for taking the original notes of Part~\ref{part:p3}.
\end{document}