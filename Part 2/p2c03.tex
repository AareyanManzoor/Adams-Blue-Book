\documentclass[../main]{subfiles}
\begin{document}
\label{sec:p2c3}

\chapter{Reformulation}

In this section we will interpret a formal group over $R$ as a group in the category of coalgebras over $R$.

The results of the previous section suggest that the algebra of formal power series $R[[x]]$, which arose \hyperref[sec:p2c1]{\S 1}, is actually the dual of the object which should be considered. Let $F$ be an $R$-module which is free on generators $\beta_0,\beta_1,..,\beta_n,...$. We make $F$ into a coalgebra over $R$ by setting
\begin{equation}
\label{eqn:p2c03.1}
\tag{3.1}
\psi\beta_k=\sum_{i+j=k}\beta_i\otimes\beta_j.
\end{equation}
The dual of $F$, given by $F^\ast=\operatorname{Hom}_R(F,R)$, is then an algebra over $R$, and it can be identified with $R[[x]]$; the pairing between $R[[x]]$ and $F$ is given by 
\begin{equation}
\label{eqn:p2c03.2}
\tag{3.2}
\Big<\sum_{i\geq 0}c_ix^i,\beta_n\Big>=c_n.    
\end{equation}
(Here the coefficients $c_i$ lie in $R$.)

The analogy with the case of a Lie group confirms that this procedure is reasonable. Instead of looking at analytic functions $\displaystyle\sum_{i\geq 0}c_ix^i$ on $G$, we look at differential operators, because functions are contravariant and differential operators are covariant. More precisely, we interpret $\beta_n$ as the differential operator $\frac{1}{n!}\frac{d^n}{dx^n}$, evaluated at $x=0$. The result of applying this operator to the analytic function $\displaystyle\sum_{i\geq 0}c_ix^i$ is indeed $c_n$. The coproduct in $F$ corresponds to Leibniz' formula
\[\frac{1}{k!}\frac{d^k}{dx^k}(fg)=\sum_{i+j=k}\Big(\frac{1}{i!}\frac{d^i}{dx^i}\Big)\Big(\frac{1}{j!}\frac{d^j}{dx^j}g\Big).\]
Since differential operators are covariant, it is reasonable that the product in $G$ should induce a product of differential operators. 

To continue, let $F$ be as above; then we can form $F\otimes_R F$, and its dual, $\operatorname{Hom}_R(F\otimes_R F, R)$, may be identified with the algebra $R[[x_1,x_2]]$. The pairing $R[[x_1,x_2]]$ and $F\otimes_R F$ is given by 
\begin{equation}
\label{eqn:p2c03.3}
\tag{3.3}    \Big<\sum_{i,j}c_{ij}x_1^ix_2^j, \beta_p\otimes\beta_q\Big> = c_{pq}.
\end{equation}
Each $R$-map
\[m_\ast:F\otimes_R F\longrightarrow F\]
induces a dual map
\[m^\ast:R[[x]]\longrightarrow R[[x_1,x_2]].\]
This induces a $1$-$1$ correspondence between maps $m_\ast$ which are filtration-preserving (in a suitable sense) and maps $m^\ast$ which are filtration-preserving; corresponding maps are given by the following formulae.
\begin{equation}
\label{eqn:p2c03.4}
\tag{3.4}
m_\ast(\beta_i\otimes\beta_j)=\sum_{k\leq i+j}a_{ij}^k\beta_k
\end{equation}
\begin{equation}
\label{eqn:p2c03.5}
\tag{3.5}
    m^\ast x^k=\sum_{i+j\geq k}a_{ij}^kx_1^ix_2^j.
\end{equation}
(Here the coefficients $a_{ij}^k$ lie in our ring $R$. The coefficients $a_{ij}^1$ are the coefficients $a_{ij}$ of section~\ref{sec:p2c1}.) The map $m^\ast$ is a map of algebras if and only if the map $m_\ast$ is a map of coalgebras. It is now easy to check that the relevant conditions on $m_\ast$ (such as associativity and commutativity) are equivalent to the corresponding conditions on $m^\ast$. The map $e:R[[x]]\longrightarrow R$, which was introduced as a unit map in \hyperref[sec:p2c1]{\S 1} and defined by $e\Big(\displaystyle \sum_{i\geq 0}c_ix^i\Big)=c_0$, now has the alternative name $\beta_0$; we take $\beta_0$; we take $\beta_0$ as our unit in $F$.

It is clear, of course, that if $m^\ast$ is a map of algebras, then $m^\ast x^k$ is determined by $m^\ast x$. So in this case, the coefficients $a_{ij}^k$ are determined by the $a_{ij}^1=a_{ij}$. For example, we easily obtain the following formula.
\begin{equation}
\label{eqn:p2c03.6}
\tag{3.6}
    a_{1j}^k = ka_{1,j+1-k}.
\end{equation}

\begin{exercise}
Obtain a formula for $a_{22}^k.$
\end{exercise}

We conclude that there is a precise equivalence between group-object structures on $R[[x]]$ in the sense of \hyperref[sec:p2c1]{\S 1}, and suitable Hopf-algebra structures on $F$. A formal group is therefore a group-object in a suitable category of coalgebras.
\end{document}