\documentclass[../main]{subfiles}
\begin{document}
\label{sec:p2c8}

\chapter{Quillen's Theorem}
By Theorem~\ref{thm:p2c05.1} we have a map \[\theta' : L \lar{} \pi_\ast(\mathrm{MU}).\] The object of this section is to prove the following results. 

\begin{theorem}[Milnor]
\label{thm:p2c08.1}
$\pi_\ast(\mathrm{MU})$ is a polynomial algebra over $\bbZ$ on generators of dimension $2, 4, 6, \ldots$. 
\end{theorem}

\begin{theorem}[Quillen]
\label{thm:p2c08.2}
The map \[\theta' : L \lar{} \pi_\ast(\mathrm{MU})\] is an isomorphism.
\end{theorem}

Following Milnor, we base our calculation of $\pi_\ast(\mathrm{MU})$ on the spectral sequence

\begin{equation}
\tag{8.3}
\label{eqn:p2c08.3}
\mathrm{Ext}_A^{s, t} (H^\ast(\mathrm{MU}; \bbZ_p), \bbZ_p) \overset{s}{\implies} \pi_{t - s}(\mathrm{MU})
\end{equation}

Here $A$ is the mod $p$ Steenrod algebra.

\begin{lemma}
\label{lem:p2c08.4}
$H^\ast(\mathrm{MU}; \bbZ_p)$ is a free module over $A/(A \beta A)$, where $A \beta A$ is the two-sided ideal generated by the Bokštein boundary $\beta = \beta_p$. 
\end{lemma}

This lemma is an absolutely standard consequence of the following facts. (i) $A/(A \beta A)$ acts freely on the Thom class $u \in H^0(\mathrm {MU}; \bbZ_p)$. (ii) $H^\ast(\mathrm{MU}, \bbZ_p)$ is a coalgebra over $A/(A \beta A)$.

Unfortunately, we do not only need to know that $H^\ast(\mathrm{MU}; \bbZ_p)$ is free over $A/(A \beta A)$; we need to know about its base; or more precisely, we need the following result. 

\begin{lemma}
\label{lem:p2c08.5}
$\mathrm{Hom}_A(H^\ast(\mathrm{MU}; \bbZ_p), \bbZ_p)$, which can be identified with the set of primitive elements in the comodule $H_\ast(\mathrm{MU}; \bbZ_p)$ is a polynomial algebra on generators of dimension $2n$ for $n > 0$, $n \ne p^f - 1$.
\end{lemma}

We prove (\ref{lem:p2c08.4}) and (\ref{lem:p2c08.5}) together, following Liulevicius. More precisely, let $A_\ast$ be the dual of $A/(A \beta A)$; it is a polynomial algebra \[\bbZ_p[\xi_1, \xi_2, \ldots, \xi_f, \ldots]\] with $\xi_f$ of dimension $2(p^f - 1)$. Let $N_\ast$ be a polynomial algebra \[\bbZ_p[x_1, x_2, \ldots, x_{p - 2}, x_p, \ldots]\] with one generator $x_i$ of dimension $2i$ whenever $i \ne p^f - 1$. Define a homomorphism \[\alpha : H_\ast(\mathrm{MU}; \bbZ_p) \lar{} N_\ast\] by \[\alpha(b_i) = \begin{cases}x_i & i \ne p^f - 1 \\ 0 & i = p^f - 1.\end{cases}\] Define a homomorphism from $H_\ast(\mathrm{MU}; \bbZ_p)$ to $A_\ast \otimes N_\ast$ by \[H_\ast(\mathrm{MU}; \bbZ_p) \lar{\psi} A_\ast \otimes H_\ast(\mathrm{MU}; \bbZ_p) \lar{1 \otimes \alpha} A_\ast \otimes N_\ast,\] where $\psi$ is the coproduct map. Make $A_\ast \otimes N_\ast$ into a comodule over $A_\ast$ by giving it the coproduct map \[A_\ast \otimes N_\ast \lar{\psi \otimes 1} A_\ast \otimes (A_\ast \otimes N_\ast).\] Then $(1 \otimes \alpha) \psi$ is a homomorphism of rings and a homomorphism of comodules over $A_\ast$.

Now, in $\mathrm{BU}(1)$ we have \[\psi \beta_{p^f} = \xi_f \otimes \beta_1 + \xi_{f-1}^p \otimes \beta_p + \ldots + 1 \otimes \beta_{p^f}.\] So in $\mathrm{MU}$ we have \[\psi \beta_{p^f - 1} = \xi_f \otimes 1 + \xi_{f - 1}^p \otimes b_{p - 1} + \ldots + 1 \otimes b_{p^f - 1}.\] We see that the map \[Q((1 \otimes \alpha) \psi) : Q(H_\ast(\mathrm{MU}; \bbZ_p)) \lar{} Q(A_\ast \otimes N_\ast)\] is given by \[Q((1 \otimes \alpha) \psi) b_i = \begin{cases}1 \otimes x_i \mod I^2 & i \ne p^f - 1 \\ \xi_f \otimes 1 \mod I^2 & i = p^f - 1.\end{cases}\]

So $Q((1 \otimes \alpha)\psi)$ is an isomorphism, and $(1 \otimes \alpha) \psi$ is an epimorphism. By counting dimensions, $(1 \otimes \alpha) \psi$ must be an isomorphism. 

Since the dual of $A_\ast \otimes N_\ast$ is free, we have proved (\ref{lem:p2c08.4}). Since the set of primitive elements in $A_\ast \otimes N_\ast$ is precisely $N_\ast$, we have proved (\ref{lem:p2c08.5}) too.

\begin{corollary}
\label{cor:p2c08.6}
In the spectral sequence \eqref{eqn:p2c08.3}, the $E_2$-term \[\mathrm{Ext}_A^{s, t} (H^\ast(\mathrm{MU}; \bbZ_p), \bbZ_p)\] is a polynomial algebra on generators $x_n$, $n = 0, 1, 2, 3, \ldots$ of bidegree 
%TODO: sort alignment
\[s = 0, \quad t = 2n, \quad (n \ne p^f - 1)\]
\[s = 1, \quad t = 2n + 1, \quad (n = p^f - 1).\]
\end{corollary}

This follows from (\ref{lem:p2c08.4}), (\ref{lem:p2c08.5}) by standard methods; see \cite{milnor2}.

It follows from (\ref{cor:p2c08.6}) that the spectral sequence \eqref{eqn:p2c08.3} has non-zero groups only in even dimensions; so the spectral sequence is trivial.

In order to deduce the required results on $\pi_\ast(\mathrm{MU})$, we need a technical lemma on the convergence of the spectral sequence \eqref{eqn:p2c08.3}.

\begin{lemma}
\label{lem:p2c08.7}
Suppose given a connected spectrum $X$, such that $\pi_r(X)$ is finitely generated for each $r$ and zero for $r < 0$. Suppose given integers $m, e$. Then there exists $s = s(m, e)$ such that any element in $\pi_m(X)$ of filtration $\ge s$ in the spectral sequence \[\mathrm{Ext}_A^{s, t}(H^\ast(X; \bbZ_p), \bbZ_p) \xRightarrow{s} \pi_{t - s}(X)\] is divisible by $p^e$ in $\pi_n(X)$.
\end{lemma}

This may be proved by the method given in my original paper \cite{adams}.

\begin{corollary}
\label{cor:p2c08.8}
\[Q_m(\pi_\ast(\mathrm{MU})) \otimes \bbZ_p = \begin{cases}\bbZ_p & \text {for } m = 2n, \, n > 0 \\ 0 & \text{otherwise}.\end{cases}\] 
\end{corollary}

\begin{proof}
When $m \ne 2n$ ($n > 0$) the result is trivial, so we assume $m = 2n$, $n > 0$. There are of course many ways of seeing that $Q_{2n}(\pi_\ast(\mathrm{MU})) \otimes \bbZ_p$ has dimension at least one over $\bbZ_p$; for example, 

\[Q_{2n}(\pi_\ast(\mathrm{MU})) \otimes \bbQ \cong Q_{2n}(H_\ast(\mathrm{MU})) \otimes \bbQ \cong \bbQ.\]

We need to prove $Q_{2n}(\pi_\ast(\mathrm{MU})) \otimes \bbZ_p$ has dimension at most $1$.

Let $t_i \in \pi_{2i}(\mathrm{MU})$ be an element whose class in the $E_2$-term is the generator $x_i$ in (\ref{cor:p2c08.6}). I claim that $Q_{2n}(\pi_\ast(\mathrm {MU})) \otimes \bbZ_p$ is generated by the image of $t_n$. In fact, let $y$ be any element in $\pi_{2n}(\mathrm {MU})$, and let $s$ be as in (\ref{lem:p2c08.7}), taking $m = 2n$, $e = 1$; then (by induction over the filtration) we can find a polynomial $q(t_0, t_1, \ldots, t_n)$ such that $y - q(t_0, t_1, \ldots, t_n)$ has filtration $\ge s$, and so \[y = q(t_0, t_1, \ldots, t_n) + p z.\] Since $\pi_0(\mathrm{MU}) = \bbZ$, the coefficient of $t_n$ (which \emph{a priori} is a polynomial in $t_0$) must be an integer $c$. We deduce that \[y = c t_n \mod I^n + p \pi_\ast(\mathrm{MU}),\] where $\displaystyle I = \sum_{i \ge 0} \pi_i(\mathrm{MU}).$ That is, $Q_{2n}(\pi_\ast(\mathrm{MU})) \otimes \bbZ_p$ is generated by the image of $t_n$. This proves (\ref{cor:p2c08.8}).
\end{proof}

\begin{corollary}
\label{cor:p2c08.9}
\[Q_m(\pi_\ast(\mathrm{MU})) \cong \begin{cases}\bbZ & \text {for } m = 2n, \, n > 0 \\ 0 & \text{otherwise.}\end{cases}\]
\end{corollary}

\begin{proof}
$Q_m(\pi_\ast(\mathrm{MU}))$ is a finitely generated abelian group; use the structure theorem for finitely generated abelian groups, plus (\ref{cor:p2c08.8}).
\end{proof}

We now consider the following diagram. 

\begin{center}
\begin{tikzcd}
L \arrow[rr, "\theta'"] \arrow[d, "\theta"']                                 &  & \pi_\ast(\mathrm{MU}) \arrow[d, "h"] \\
{{\mathbb Z}[b_1, b_2, \ldots, b_n, \ldots]} \arrow[rr, equals] &  & H_\ast(\mathrm{MU})                 
\end{tikzcd}
\end{center}

Here $\theta$ has been carefully defined so that the diagram is commutative, as we see by comparing (\ref{cor:p2c06.6}) with \eqref{eqn:p2c07.2}, \eqref{eqn:p2c07.7}. The behaviour of $\theta$ has been studied in \hyperref[sec:p2c7]{\S 7}. 

\begin{lemma}
\label{lem:p2c08.10}
The image of \[Q_{2n}(h) : Q_{2n}(\pi_\ast(\mathrm{MU})) \lar{} Q_{2n}(H_\ast(\mathrm{MU}))\] is the same as the image of $Q_{2n}(\theta)$ (which has described in (\ref{lem:p2c07.9})). 
\end{lemma}

\begin{proof}
It is clear that $\mathrm{Im} \, Q_{2n}(\theta) \subset \mathrm{Im} \, Q_{2n}(h)$; we have to prove $\mathrm{Im} \, Q_{2n}(h) \subset \mathrm{Im} \, Q_{2n}(\theta)$. If $n + 1 \ne p^f$ there is nothing to prove. If $n + 1 = p^f$, consider the canonical map \[\mathrm{MU} \lar{} H \lar{} H \bbZ_p;\] call it $g$. The induced homomorphism \[q_\ast : H_\ast(\mathrm{MU}) \lar{} (H \bbZ_p)_\ast (H \bbZ_p)\] clearly annihilates the image of $\pi_{2n}(\mathrm{MU})$. On the other hand, it carries $b_n$ into the Milnor generator $\xi_f$ in $(H \bbZ_p)_\ast (H \bbZ_p)$ (since both come from $\mathrm{MU}(1) = \mathrm{BU}(1)$). The class $\xi_f$ remains non-zero when we pass to $Q_{2n}(H \bbZ_p)_\ast (H \bbZ_p) \cong \bbZ_p$. So the image of $Q_{2n}(h)$ consists at most of the multiples of $p[b_n]$. This proves (\ref{lem:p2c08.10}).
\end{proof}

\begin{exercise}
See if you can refrain from translating this proof into cohomology. 
\end{exercise}

We proceed to prove (\ref{thm:p2c08.1}) and (\ref{thm:p2c08.2}). Recall our diagram.

\begin{center}
\begin{tikzcd}
L \arrow[rr, "\theta'"] \arrow[d, "\theta"']                                 &  & \pi_\ast(\mathrm{MU}) \arrow[d, "h"] \\
{{\mathbb Z}[b_1, b_2, \ldots, b_n, \ldots]} \arrow[rr, equals] &  & H_\ast(\mathrm{MU})                 
\end{tikzcd}
\end{center}
%Corrected extraneous bracket here
It follows from (\ref{cor:p2c08.9}) and (\ref{lem:p2c08.10}) that \[Q_{2n}(h) : Q_{2n}(\pi_\ast(\mathrm{MU})) \lar{} \mathrm{Im} \, Q_{2n}(\theta)\] is iso. Using (\ref{cor:p2c07.14}), we see that \[Q_{2n}(\theta') : Q_{2n}(L) \lar{} Q_{2n}(\pi_\ast(\mathrm{MU}))\] is iso. But by (\ref{thm:p2c07.8}), the map $\theta = h \theta'$ is mono; so $\theta'$ is mono, and $\theta$ is an isomorphism. This proves (\ref{thm:p2c08.2}), and (\ref{thm:p2c08.1}) follows from (\ref{thm:p2c07.1}). 

Taking a last look at our diagram, we conclude that the homomorphism $\theta$ studied in \S\plscite{7} was up, to isomorphism, the Hurewicz homomorphism \[h : \pi_\ast(\mathrm{MU}) \lar{} H_\ast(\mathrm{MU}).\] 

\begin{corollary}
\label{cor:p2c08.11}
The Hurewicz homomorphism \[h : \pi_\ast(\mathrm{MU}) \lar{} H_\ast(\mathrm{MU})\] is a monomorphism. 
\end{corollary}

\begin{exercise}
Deduce (\ref{thm:p2c08.1}) directly from (\ref{cor:p2c08.6}).
\end{exercise}
\end{document}