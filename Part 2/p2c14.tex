\documentclass[../main]{subfiles}

\begin{document}
\label{sec:p2c14}
%wordslinger

\chapter{The Hattori-Stong theorem}

In this section I will present a slight reformulation of the result
of Hattori and Stong. (Stong proved it first, but his name creeps to the
back for reasons of euphony--it brings a phrase or sentence to such a
resounding end.) This reformulation has been used by L. Smith \cite{smith}. 

Recall from \cite[Lecture 3]{adams3} that for suitable spectra $E$, such as $E=K$, $E_\ast(X)$ is a comodule over the Hopf algebra $E_\ast(E)$. We say that an element in a comodule is \emph{primitive}\index{primitive element of a comodule} if $\psi x = 1\otimes x$; we write $\mathrm{PE}_\ast(X)$ for the subgroup of primitive elements in $E_\ast(X)$. One can see directly from the definition of $\psi$ that the Hurewicz homomorphism in $E$-homology, \[h:\pi_\ast(X)\lra E_\ast(X)\] maps into $\mathrm{PE}_\ast(X)$. 
\begin{theorem}[after Stong \cite{stong} and Hattori \cite{hattori}]
\label{thm:p2c14.1}
The Hurewicz homomorphism in $K$-homology gives an isomorphism \[h:\pi_\ast(\mathrm{MU})\cong\mathrm{PK}_\ast(\mathrm{MU}).\]
\end{theorem}
\emph{Remark.} As soon as one knows that $\pi_\ast(\mathrm{MU})$ is torsion-free, it is easy to show that the Hurewicz homomorphism is a monomorphism. For example, consider the following commutative diagram.
\begin{center}
\begin{tikzcd}
	{\pi_\ast(\mathrm{MU})} &&& {K_\ast(\mathrm{MU})} \\
	\\
	{\pi_\ast(\mathrm{MU})\otimes\mathbb{Q}} &&& {K_\ast(\mathrm{MU})\otimes\mathbb{Q}} \\
	& {}
	\arrow["h", from=1-1, to=1-4]
	\arrow[from=1-4, to=3-4]
	\arrow[from=1-1, to=3-1]
	\arrow["{h\otimes 1}", from=3-1, to=3-4]
\end{tikzcd}
\end{center}
We have $K_\ast(\mathrm{MU})\otimes\mathbb{Q}\cong\pi_\ast(K)\otimes\pi_\ast(\mathrm{MU})\otimes\mathbb{Q}$; so the bottom horizontal map and the left-hand vertical map are both monomorphisms.

The essential content of the theorem, then, is that it identifies the images of $h$. 
\begin{proof}[Proof of \ref{thm:p2c14.1}]
For lack of time in writing out these notes to work out a direct proof, I will deduce this result from the formulation given by Hattori. (After all, Hattori's proof is very elegant.) Hattori
proves precisely that if $x\in K_\ast(\mathrm{MU}$ and $nx\in \mathrm{Im}(h)$ for some integer
$n\neq 0$, then $x \in\mathrm{Im}(h)$. It is rather easy to see that any primitive in $K_\ast(\mathrm{MU}) \otimes \mathbb{Q}$ lies in the image of $h\otimes 1$. So suppose $x \in \mathrm{PK}_\ast(\mathrm{MU})$; then
by preceding sentence, $x$ lies in $\mathrm{Im}(h\otimes 1)$; that is, for some integer $n\neq 0$ we have $nx \in\mathrm{Im}(h)$. So by Hattori's form of the result, $x \in\mathrm{Im}(h)$,
This proves \eqref{thm:p2c14.1}. 
\end{proof}
\emph{Exercise.} Deduce Hattori's form of the result from \eqref{thm:p2c14.1}.
\end{document}