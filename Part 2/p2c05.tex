\documentclass[../main]{subfiles}
\begin{document}
\label{sec:p2c5}

\chapter{Lazard's Universal Ring}

In this section we introduce Lazard's universal ring. Following Fr\"{o}hlich \cite{frolich}, we call this ring $L$ (for Lazard).

\begin{theorem}
\label{thm:p2c05.1}
There is a commutative ring $L$ with unit, and a commutative formal product $\mu^L$ defined over $L$, such that for any commutative ring $R$ with unit and any commutative formal product $\mu^R$ defined over $R$ there is one and only one homomorphism $\theta:L\longrightarrow R$ such that \[\theta_\ast\mu^L = \mu^R.\]
\end{theorem}
\begin{proof}
We define $L$ by generators and relations; that is, we define $L$ as the quotient of a polynomial ring $F$ by an ideal $I$. Take formal symbols $a_{ij}$ for $i\ge 1$, $j\ge 1$ and set \[P = \bbZ [a_{11}, a_{12}, a_{21}, \ldots, a_{ij}, \ldots].\]
Form the formal power series \begin{equation}
\label{eqn:p2c05.2}
\tag{5.2}
\mu(x,y) = x+y+\sum_{i,j\ge 1} a_{ij} x^i y^j
\end{equation}
and set 
\begin{equation}
\label{eqn:p2c05.3}
\tag{5.3}
\mu(x, \mu(y,z)) - \mu(\mu(x,y),z) = \sum_{i,j,k} b_{ijk} x^i y^j z^k
\end{equation}
Then each coefficient $b_{ijk}$ is a well-defined polynomial in the $a_{ij}$. Take $I$ to be the ideal in $P$ generated by the elements $b_{ijk}$ and $a_{ij} - a_{ji}$. It is trivial to check that $L=P/I$ has the required properties. 
\end{proof}
We note that we can make $L$ into a graded ring if we wish. In fact, we assign to $x,y$ and $\mu(x,y)$ the degree $-2$; then $a_{ij}$ has degree $2(i+j-1)$, and $b_{ijk}$ is a homogeneous polynomial of degree $2(i+j+k-1)$. It follows that $I$ is a graded ideal and $P/I$ is a graded ring.

We note the structure of $L$ is in principle computable. For example, 
\begin{align*}
L_0 &\cong \bbZ,\text{ generated by }1\\
L_2 &\cong \bbZ,\text{ generated by }a_{11}\\
L_4 &\cong \bbZ\oplus\bbZ,\text{ generated by }a^2_{11}\text{ and }a_{12}\\
L_6 &\cong \bbZ\oplus\bbZ\oplus\bbZ,\text{ generated by }a^3_{11}, a_{11}, a_{12}\text{ and }a_{22}-a_{13}.
\end{align*}
\begin{exercise}
Obtain the relation which allows one to write $a_{22}$ and $a_{31}$ in terms of the relation given.
\end{exercise}
The structure of $L$ will be described in more detail in our next algebraic section, \hyperref[sec:p2c7]{\S 7}. 

In order to obtain the structure of $L$, we use algebraic arguments which are openly obtained by analogy with the situation in algebraic topology. 
\end{document}