\documentclass[../main]{subfiles}
\begin{document}
\label{sec:p2c0}

\chapter{Introduction}


These notes derive from a series of lectures which I gave in Chicago in April 1970. It is a pleasure to thank my hosts for an enjoyable and stimulating visit.

In \S\S1-8, I have tried to give a connected account, beginning from first principles and working up to Milnor's calculation of $\pi_\ast(\mathrm{MU})$ \eqref{thm:p2c08.1} and Quillen's theorem that $\pi_\ast(\mathrm{MU})$ is isomorphic to Lazard's universal ring $L$ \eqref{thm:p2c08.2}. The structure of $L$ is obtained from first principles \eqref{thm:p2c07.1}. This is done by relating the notion of a formal group to the notion of a Hopf algebra. The material has been so arranged that algebraists who are interested in the subject can obtain a fairly self-contained account by reading \S\S\hyperref[sec:p2c1]{1}, \hyperref[sec:p2c3]{3}, \hyperref[sec:p2c5]{5}, \hyperref[sec:p2c7]{7}.

The remaining sections deal with related matters, In \cite[Lecture 3]{adams3}, I have shown that for suitable spectra $E$, $E_\ast(E)$ can be given the structure of a Hopf algebra analogous to the dual of the Steenrod algebra. The structure of this Hopf algebra is described for the spectrum $\mathrm{MU}$ in \hyperref[sec:p2c11]{\S 11}, for the $\mathrm{BU}$-spectrum in \hyperref[sec:p2c13]{\S 13}, and for the Brown-Peterson spectrum in \hyperref[sec:p2c16]{\S 16}. Sections 15 and 16 are devoted to Quillen's work on the Brown-Peterson spectrum \cite{brownpeterson}. \hyperref[sec:p2c14]{\S 14} is devoted to the Hattori-Stong theorem. 
\end{document}